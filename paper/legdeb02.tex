\documentclass[letter,12pt]{article}
\usepackage[letterpaper,right=1in,left=1in,top=1in,bottom=1in]{geometry}
\usepackage{setspace}

\usepackage[utf8]{inputenc}   % allows input of special characters from keyboard (input encoding)
\usepackage[T1]{fontenc}      % what fonts to use when printing characters       (output encoding)
\usepackage{amsmath}          % facilitates writing math formulas and improves the typographical quality of their output
\usepackage[hyphens]{url}     % adds line breaks to long urls
\usepackage[pdftex]{graphicx} % enhanced support for graphics
\usepackage{tikz}             % Easier syntax to draw pgf files (invokes pgf automatically)
\usetikzlibrary{arrows}

\usepackage{mathptmx}           % set font type to Times
\usepackage[scaled=.90]{helvet} % set font type to Times (Helvetica for some special characters)
\usepackage{courier}            % set font type to Times (Courier for other special characters)

\usepackage[longnamesfirst, sort]{natbib}\bibpunct[]{(}{)}{,}{a}{}{;} % handles biblio and references 

\usepackage{rotating}         % sideway tables and figures that take a full page
\usepackage{caption}          % allows multipage figures and tables with same caption (\ContinuedFloat)

\usepackage{dcolumn}          % needed for apsrtable and stargazer tables from R to compile
\usepackage{arydshln}         % dashed lines in tables (hdashline, cdashline{3-4}, 
                              %see http://tex.stackexchange.com/questions/20140/can-a-table-include-a-horizontal-dashed-line)
                              % must be loaded AFTER dcolumn, 
                              %see http://tex.stackexchange.com/questions/12672/which-tabular-packages-do-which-tasks-and-which-packages-conflict


\newcommand{\mc}{\multicolumn}

\usepackage{epigraph}          % format epigraphs

%% TO ADD NOTES IN TEXT, PUT % BEFORE THE ONE YOU WANT DISABLED
\usepackage[disable]{todonotes}                            % no show
%\usepackage[colorinlistoftodos, textsize=small]{todonotes} % show notes
\newcommand{\emm}[1]{\todo[color=red!15, inline]{\textbf{Eric:} #1}}
\newcommand{\vp}[1]{\todo[color=green!15, inline]{\textbf{Vale:} #1}}
\newcommand{\ges}[1]{\todo[color=blue!15, inline]{\textbf{Ges:} #1}}

\usepackage{xr} % allows cross-ref to other file
\externaldocument{urge15appendix}

%% %for submission: sends figs, tables, and footnotes to last pages
%% \RequirePackage[nomarkers,nolists]{endfloat}     % sends tables and figures to the end
%% \RequirePackage{endnotes}                        % turns fn into endnotes; place \listofendnotes where you want 
%%                                                  %the endnotes to appear (it must be after the last endnote).
%% \let\footnote=\endnote
%% \newcommand{\listofendnotes}{
%%    \begingroup
%%    \parindent 0pt
%%    \parskip 2ex
%%    \def\enotesize{\normalsize}
%%    \theendnotes
%%    \endgroup
%% }
%% 
%% % for submission: drop page numbers when producing title page
%% \pagenumbering{gobble} % Remove page numbers (and reset to 1)
%% \pagenumbering{arabic}% Arabic page numbers (and reset to 1)


\setcitestyle{citesep={;}}

\usepackage{listings}

\begin{document}

\title{Floor access in Mexico's Cámara de Diputados\thanks{Eric Magar received financial support from the Asociaci\'on Mexicana de Cultura \textsc{a.c.}\ and \textsc{conacyt}'s Sistema Nacional de Investigadores. I thank Fernando Rodríguez Doval, Lupita Vargas Vargas, and one former lawmaker who wished anonymity for shedding light on some parties' internal rules of debate in the period. I am grateful to Ana Lucía Enríquez, Eugenio Solís Flores, Sonia Kuri, and Vidal Mendoza for research assistance. The author is responsible for mistakes and shortcomings in the study. Data and supporting materials necessary to reproduce the quantitative analysis are available for download at \url{https://github.com/emagar/legdeb}.}}
\author{Eric Magar \\ Instituto Tecnológico Autónomo de México}
\date{\today}
\maketitle

%\newpage

\begin{abstract}
\noindent This chapter describes the institutions of legislative debate in the Mexican Cámara de Diputados and assesses predictors of floor participation. Multiple regression models are fit on more than twenty-three thousand speeches between 2006 and 2020. They shows that majority party members get privileged floor access, in both the number of speeches delivered and their word-length, even after accounting for the negative effect of party size. Other status indicators, such as committee chairs, party leaders, and seniority, have more modest but also positive effects in debate. Women speak more than men. And the removal of single-term limits in 2018, which tend to personalize elections, associate with a significant surge in floor participation. 
\newline
\newline
\textbf{Keywords}: Floor debate, speech, Congress, presidential democracy, Mexico
\newline
\newline
\textbf{Word count}: 6469 (including abstract and references)
\end{abstract}

\newpage

\doublespacing

\section{Introduction} % [max 500 words]
%% -describe what the chapter is about, key findings, and distinctive features of the country

Legislative studies are a relatively young field of Mexican politics. Its growth is remarkable and much has been learned in a wide array of areas. These include candidate selection \citep{ascencio.kerevel.cand-sel-beh.2021}; congressional campaigns \citep{langston.nd}; redistricting \citep{magar.altman.mcd.trelles2016pg}; vote trading \citep{lopez.lara.aldf2013}; pork barreling \citep{kerevelPork2015}; instability \citep{heller.weldon.2003}; roll call voting \citep{cantuDesposatoMagar.MxRcv.2014}; federal influences \citep{rosas.langston.2011}; constitutional amendment \citep{casar.marvan2014book}; executive success \citep{bejarQuienLegisla2012}; divided government \citep{casarSinMay2013}; the budget process \citep{weldon.2002}; executive predominance \citep{weldon.1997}; and party discipline \citep{tellez-del-rio.2018}, among others.

But there is no scholarship on legislative debate in sight. Other than brief and general mentions to the subject, I could find no systematic study of floor access. This chapter aims to start filling the void by describing the institutional setting of debate in the Cámara de Diputados and performing a systematic examination of the determinants of floor participation.

A disconnect appears between formal and informal institutions. Formal rules decentralize agenda power by granting members broad rights of recognition to take the floor and deliver speeches. Informal rules channel debate through legislative parties, leaders managing participation in centralized fashion. Unlike the U.K. Parliament, where delegation to the cabinet annuls most private members' formal rights \citep{cox.1987}, the Cámara appears to belong in a middle ground, where the U.S. Senate probably also lies \citep{denhartog.monroe.SenateBook.2011}.

Focus is on the Cámara de Diputados of the bicameral Congress. The chambers have symmetric powers over most legislation, but the Senate is excluded from adoption of the annual budget, and I leave it out. Moreover, due to time constraints, I further narrow the focus to three out of eight Cámara terms since the advent of competitive politics in Mexico. I examine the 60th Legislature (2006-09), the 62nd (2012-15), and the 64th (2018-21) up to the end of the second ordinary year---enough to investigate how the removal of term limits affects debate.

The chapter is organized thus. Section 2 describes political institutions, the party system, and major changes to both. Section 3 describes the institutional setting of legislative debate in the Cámara. It identifies key players, the structure of debate, recognition-granting motions, and how party discipline works as a substitute to centralized agenda power. Section 4 performs data analysis. Multiple regression models reveal the mutual influences of parties and individual rights in the number of speeches deputies make and their length. Section 5 discusses minority rights in the context of Mexican politics, and concludes. 

\section{Institutional and party system background} % [ca 500-1000 words]

%% In this section, you need to address the basic institutional design of your country.
%% Please cover, at least, the following points:

%% 1. Describe the institutional system: executive-legislative relations; balance of power between executive and legislative branches. Does the executive dominate?

%% 2. Parties and party system: describe the most important parties in the window of observation of the chapter. Discuss party system features. Are there dominant parties? Are there extreme left or right parties? Are the latter considered outsider parties?

%% 3. Discuss the organization of parliamentary party groups/legislative parties. The balance of power between frontbenchers (leadership) and backbenchers.

%% 4. Electoral system: discuss personal vote-seeking incentives. Frame this in a way that answers the questions: is the electoral system party-centered or candidate-centered? Discuss existing empirical work about your country that deals with home style. Discuss the ballot structure and how it allows voters to reward/punish individual legislators. Conclude point by referring to the importance of the ‘party brand’ as an electoral asset? Is the party brand important for vote-seeking (e.g. in CLPR, such as Portugal, Spain or Norway) or is the party brand of less importance (e.g. in Irish STV)?

%% 5. How important is party unity/cohesion for electoral success? Do institutional rules make party leaders value unity or can they allow MPs to dissent?

\subsection{Executive-legislative relations}

Mexico is a presidential democracy. For most of the 20th century a hegemonic party, the Institutional Revolutionary Party (PRI), held the strings of political influence in a tight grip nationwide. While the PRI's electoral fortunes suffered from societal change and from formidable economic setbacks in the 1980s, it was not until 1997 that competitive politics became the norm \citep{scott.1959,cosio.villegas.1981,molinar.1991a,cornelius.1996}. For the first time in over six decades, the PRI lost control of the lower chamber of Congress in that year's midterm election. Then in 2000 the country's long-standing right-of-center opposition, the National Action Party (PAN) beat the PRI in the presidential race.  

With democracy came two decades of divided government. The executive's control of the legislative process ended abruptly, inaugurating relative balance between the branches \citep{weldon.1997,lujambio.segl.2000}. The president retained a prominent role in lawmaking, but genuine negotiation with the opposition was required to get things done \citep{casarSinMay2013,bejarQuienLegisla2012}. 

The competitive era had a system with three major parties and a handful of small opportunistic parties \citep[see][]{moreno.decisElec.2009,diaz-estevez-magaloni-Poverty-book.2016}. Majors included the PAN, the PRI, and a left-of-center Democratic Revolutionary Party (PRD). Competition was mostly between the PRI and another major at the local level. The PRI retained strongholds from its hegemonic era in rural Mexico, but neither party had particularly strong ties to social groups. Parties would have to rebuild a clientelistic coalition from near scratch at every electoral campaign.

\begin{table}
\begin{scriptsize}
\begin{verbatim}
|----------------------------+---------+---------+---------|
|                            |    60th |    62nd |    64th |
|                            | 2006-09 | 2012-15 | 2018-21 |
| Party                      |       % |       % |       % |
|----------------------------+---------+---------+---------|
| pan                        |      41 |      23 |      16 |
| pri                        |      21 |      43 |       9 |
| prd                        |      25 |      20 |       4 |
| morena                     |         |         |      51 |
| opportunistic w/ president |         |       8 |      14 |
| other opportunistic        |      13 |       7 |       6 |
|----------------------------+---------+---------+---------|
| Total                    % |     100 |     100 |     100 |
| N                          |     500 |     500 |     500 |
|----------------------------+---------+---------+---------|
| President's party          |     pan |     pri |  morena |
|----------------------------+---------+---------+---------|
\end{verbatim}
\end{scriptsize}
\caption{Parties in three Legislatures of the Cámara de Diputados}\label{T:seats}
\end{table}

%% |              |  60 |  62 |     64 |
%% |              |   N |   N |      N |
%% |--------------+-----+-----+--------+
%% | pan          | 207 | 114 |     79 |
%% | pri          | 104 | 213 |     47 |
%% | prd          | 125 | 102 |     20 |
%% | morena       |     |     |    255 |
%% | opport w pdt |     |  38 |     69 |
%% | oth opport   |  64 |  33 |     30 |
%% |--------------+-----+-----+--------+
%% | Total        | 500 | 500 |    500 |

The three-party system came crashing down in the critical election of 2018. After decades of infighting the left finally split. The faction loyal to Andrés Manuel López Obrador, known as AMLO, successfully launched the National Regeneration Movement (Morena), a new party, overcoming formidable entry barriers \citep{magar.2007ref.2015}. This feat paved his way to winning the presidency by a landslide. Riding AMLO's coattails, Morena acquired majority status in the Cámara, the first instance of single-party unified government since democratization. Inclusion of the incomplete 64th Legislature brings unified government to the study (data runs up to March 19th, 2020 which marks the end of the second ordinary year). The other terms offer perspective: a minority president in the 60th, an informal coalition with opportunistic parties in the 62nd. 


%One extraordinary event was the critical election of 2018, which marks the demise of the three-party system that brought democracy. A faction led by Andrés Manuel López Obrador, known as AMLO, split from the left, forming the National Regeneration Movement (Morena), a new party, winning the presidency in 2018. Riding AMLO's coattails, Morena acquired majority status in the Cámara. Table \ref{T:seats} reports legislative parties in the three terms examined.

\subsection{Legislative parties}
\singlespacing
\epigraph{"At the end of the day, the chambers are a mandarinate where the few decide for all."}%
{---Former lawmaker from the left, interviewed on condition of anonymity, June 17th, 2020}
\doublespacing

Weak parties in the electorate lie in sharp contrast to their strength in Congress, which stems from electoral rules. The formula is mixed member plurality---three-hundred deputies elected by first-past-the-post in single member districts (SMDs), two-hundred more by closed-list proportional representation (PR) every three years \citep{weldonMixedMemberSys2001}. All seats are contested in races concurrent with the presidential election, then again at the presidential midterm.

A key feature are single-term limits, which the constitution set on every elected officeholder. Political ambition could only be progressive \citep{schlesinger.1966}. On top of this is highly centralized ballot access: national and state party leaders control most nominations \citep{rosas.langston.2011,langston.2008}.\footnote{Reliance in primaries for SMD candidate selection, mostly by the PAN \citep{ascencio.kerevel.cand-sel-beh.2021}, on occasions by the PRI \citep{poire.phd.2002}, opens room for exceptions to centralized ballot access.} This institutional combination not just removes personal vote incentives \citep{carey.shugart.1995}, it rewards discipline to party leaders. This, we see below, plays a fundamental role in floor access. 

In a surprising recent development, single-term limits were eliminated for selected offices, including federal deputies. The 2021 midterm election will be the first since the 1930s where incumbents are allowed on the ballot \citep[see][ for details]{magarInstReel.2017}. This should introduce a degree of personal vote seeking among a subset of deputies with static ambition. While reformers further centralized nominations by keeping term limits in place for party switchers, this might not fully reign in competitive incumbents. Parties removing quality candidates---such as previous winners of elected office \citep{jacobson.1997}, dynastic candidates \citep{enriquez-dinastias2018itam}, and what \citet{zallerprizeFighters} calls "prize fighters"---in order to secure nomination of docile newbies, risk losing those districts. The 64th Legislature, despite partial data, allows examination of the effect that static ambition has on debate.

\section{The institutional setting of legislative debate in the Cámara} % [ca 1500 words]

%% In this section, we need to set the rules of the game for legislative debates. As we have seen in our workshop, there is a high heterogeneity in the institutional setting of legislative debates. Producing a thorough discussion of formal and informal rules across a wide-range of countries is a valuable contribution of the volume to the discipline. Please try to answer the following questions in your country chapter:

%% 1. Who can allocate speaking time to individual MPs? Who controls the legislative agenda – government, parties, MPs, some legislative structure specifically in charge with agenda-setting? In answering this question, you should aim to discuss both formal rules (constitutional rules, rules of procedure), party internal rules (if they exist), and informal rules.
%% 2. Do individual MPs have a guaranteed right to participate in a debate without party leadership permission? If yes, how much time and in which type of debate? 
%% 3. How is the debate structured? What are the rules of engagement? Do MPs engage in back-and-forth talk? By contrast, do MPs deliver a debate without interruptions?
%% 4. Do particular subsets of MPs (e.g. spokesperson or committee members) have formal rights that entitle them to take the floor?
%% 5. What are the existing types of debates? Please select the 5 most important types of debates and produce a table where you cover the name of the debate (in English and the native language of the country), its goals, particular organization rules to that kind of debate, and the total number of minutes that rules assign to each debate.

%% To conclude this section, please classify the country along Proksch and Slapin’s (2015) contribution. The authors suggest that there are two extreme poles, depending on rules of procedure and electoral system organization. As described by Proksch and Slapin (2015: 96), in some systems, individual MPs are ‘guaranteed access to speaking time’, and ‘backbenchers are granted equal time as party leaders’. In other systems, the rules severely restrict individuals’ access to the floor and parties draft speakers lists, which gives party leaders much more control on this question. A number of countries fall, however, somewhere between these two extreme categories, giving some opportunity for individual MPs to access the floor, but favoring party lists. Please classify your country accordingly. If your country was included in Proksch and Slapin’s study, please refer to this classification when discussing your country’s institutional setting.

An overview of the structure of legislative debate shows members who have abdicated most formal speech rights to the party. The Cámara's Rules \citep{reglamentoDipMx.2019} has prescriptions for debate, the Organic Law \citep{loceum.2019} for congressional organization.

\citet{casar.agsetting.2016} characterizes debate as party-centered: "[governing] bodies have the power ... to conduct floor debates, including assigning turns and time to speakers" (p. 154). This, we will see, comes from party discipline, because formal institutions establish individual member rights to be recognized by the presiding officer.

%\citet{casar.agsetting.2016} examination of agenda setting puts the focus on results (passage of legislation). Her mention to debate characterizes it as party-centered: "[governing] bodies have the power ... to conduct floor debates, including assigning turns and time to speakers" (p. 154). This, we will see, is not in alignment with formal institutions, which establish individual member rights to be recognized by the presiding officer.

Proksch and Slapin's \citeyearpar{proksch-slapin2015book} scheme, used across chapters in this volume, compares assemblies according to how members gain access to take the floor in order to deliver speeches (p. 79). They posit a continuum connecting two extremes: party-controlled and individual member-controlled floor access. Formal rules place the Cámara towards the individual member-controlled access limit of the continuum; but partisan rules pull it towards the party-controlled access side. The removal of single term limits ought to make this tension between formal and de facto institutions harder to manage for all parties.  

  \subsection{The boards}
There are two key actors in the legislative process, the Junta and the Mesa. The *Junta de Coordinación Política* is the Cámara's top decision-making organ. The leaders of all parties with no fewer than five deputies are represented. The majority leader presides the Junta throughout the term. In the absence of a majority party, however, the leaders of the top-three seat holding parties preside the Junta, alternating one year each. The Junta appoints and replaces committee members, prepares each session's order of the day (/orden del día/), and in general reaches and enforces party leader agreements. It decides by majority rule, with members' votes weighted relative to group sizes in the plenary. So majority status is crucial to control the Junta \citep[cf.][]{cox.mccubbins.2005}.

%One is the *Cámara president*, an officer similar to the Speaker in the UK House of Commons. Diputados are expected to address the chamber president, who is responsible for keeping debate orderly and within chamber and congressional rules. Other officers are secretarios, responsible for formalities such as reading bills, committee reports, or other motions presented to the plenary for consideration, announcing the result of roll call votes, an so forth.  


The *Mesa Directiva* is the chamber's steering board. The Mesa chair is the Cámara president ex-officio. The Mesa makeup has consensual traits, regardless of there being a majority party or not. It is elected yearly by two-thirds supermajority of Cámara members from candidates proposed by the Junta. While Mesa members can reelect, the chair must rotate between the top-three seat-holding parties, one year each. 

%The president recognizes speechmakers and presides over Cámara debates. The Mesa includes deputy presidents and at least one secretary from each parliamentary group. The Mesa follows the session's agenda in the Day's Order (/Orden del Día/) that is mostly set by the Junta. 

% I use the terms party and parliamentary group interchangeably, but groups are in fact more restrictive. A group (/grupo parlamentario/) needs no fewer than five deputies to earn and retain Junta representation. Moreover, groups can only form at the term's outset. Members can freely leave one group and may join another existing group at any time, with immediate effects in vote weights. But defectors cannot form new groups---no doubt raising the cost of lone turncoats. 

Agenda control is frail. First, every committee report is guaranteed floor consideration and must be included in the order. If committees were adequate agents of the Junta majority, they might serve as gatekeepers by denying unwanted bills a report. But the Junta is required to distribute committee chairs (as well as committee seats) proportionally among the parties, so some committees are bound to be preference outliers.

Second, the open rule is the default for bill consideration in the floor. Debate takes place in two stages. The entire bill is first examined /en lo general/, then articles are considered individually for amendment or deletion /en lo particular/ \citep[see][]{heller.weldon.nd}. Members can always reserve articles for deletion or amendment, denying the Junta a useful procedural tool common in other assemblies: the closed rule \citep[eg.,][]{cox.2006,weingast.1992,magar-palanza-sin-Pdt-fast-track-chile-2021jop}.

Third, and most relevant, speakers can self-select. Individual members are entitled to take the floor when recognized by the presiding officer, for a duration set by rules or by party agreements. Party leaders allocate speaking time to a list of speakers but cannot preclude others from adding their names to that list, making debate resemble first-come-first-serve once parties have spoken. 

%%Formally, deputies enjoy equal rights and duties. With respect to deliberation, all are entitled to take the floor when recognized by the presiding officer, for a duration set by rules or by party agreements. However, as in assemblies worldwide, members with status tend to have many more rights, the rest many more duties. What defines status changes from one assembly to the next \citep{cox.2006}. In the Cámara, party leaders in general, and majority party leaders in particular sit atop the status pyramid. But when it comes to speech, individual members retain debate rights that set the basis for minority rights. 

  \subsection{The structure of debate}
Rules set limits for different kinds of debate summarized in Table \ref{T:types}. The first entry refers drafters of new legislation, who who get first recognition to take the floor in order to persuade fellow lawmakers. The time limit is ten minutes when the draft is a new law, five minutes when it amends existing statutes. Deputies who wish to debate then get five minutes each. Bills that cannot be presented before the session ends migrate to the next day's order upon author's request /viva voce/ (otherwise they are referred to committee.) The rightmost columns report who selects the speaker---self-selection by drafting a bill, in this case---and who, if anyone, can veto the speaker's recognition---no one here. 

\begin{table}
  \begin{scriptsize}
    \begin{verbatim}
|----------------------------------------+---------------+---------+------------+------------|
| Debate type (in Spanish)               | Goal          | Durat.  | Selector   | Veto       |
|----------------------------------------+---------------+---------+------------+------------|
| 1. Introduce legislation (iniciativa)  | Author        |         |            |            |
| - a new law                            | presents      | - 10'   | - member   | - no       |
| - amend a law                          | the bill      | - 5'    | - member   | - no       |
|----------------------------------------+---------------+---------+------------+------------|
| 2. Committee report (dictamen)         | Move          |         |            |            |
| - Debate en lo general vs SQ, chair    | for floor     | - 10'   | - comm.maj | - pres.^1  |
| -   "             "      "   others    | consideration | - 5'    | - members  | - pres.^1  |
| - Amendments (debate en lo particular) |               | - 5'    | - members  | - no       |
| - negative report                      |               | - 3'    | - comm.maj | - pres.^1  |
|----------------------------------------+---------------+---------+------------+------------|
| 3. Resolutions (puntos de acuerdo)     | Position      |         |            |            |
| - standard, author                     | taking        | - 10'   | - member   | - comm.maj |
| - urgent, author (obvia resolución)    |               | - 5'    | - Junta    | - floor    |
| - other speakers                       |               | - 3'    | - party    | - no       |
|----------------------------------------+---------------+---------+------------+------------|
| 4. Current events (agenda política)    | Position      | < 2hrs  |            |            |
| - Junta proponent                      | taking        | - 10'   | - Junta    | - no       |
| - other speakers                       |               | - 5'    | - member   | - no       |
|----------------------------------------+---------------+---------+------------+------------|
| ^1 = President can delay/prevent speech by granting recommit.                                |
|--------------------------------------------------------------------------------------------|
\end{verbatim}
  \end{scriptsize}
\caption{Types of debate}\label{T:types}
\end{table}

Other speech types grant right of first recognition differently. Debate /en lo general/ grants it to the reporting committee chairperson or designated handler of the report for ten minutes (fifteen in constitutional amendments). The Cámara president can delay debate by recommitting the bill---and possibly prevent it if the committee kills the bill. /En lo particular/ amendments and Cámara resolutions grant it to the proposing member. 

Party-appointed speakers get five minutes each, in reverse-size order, after the first /en lo general/ speech. Then members who request it then get five minutes each, the president arranging them in rounds, one for one against. After six such rounds, the floor can either proceed to vote, or continue with blocks of three such rounds. When the report deals with issues of great interest, debate can go on for several hours.

%Floor (108) must approve member requests to consider an article separately from general discussion (deletion amendment?)

%Votos en lo particular = another window for dissenters. Minority report. Minority reports get floor consideration only when floor rejects the committee report. 

%Discusión en lo particular = introduce amendments to committee report (who can do it? who can block? 110 silence suggests that any member can do it --- open rule unless rules are suspended?) Proposer gets 5 minute speech, then for/againsts get 5 minute each.

Cámara resolutions (/proposiciones con punto de acuerdo/) are tailor-made for members' position-taking needs, conditional on party leader support. If adopted, resolutions become the opinion of the chamber on some specific issue. But they require urgent status (/urgente u obvia resolución/) in order to avoid committee referral and move directly to the floor; only the Junta can request that the floor grants urgent status to at most two resolutions per session. If granted, the proposer takes the floor for five minutes. Parties then appoint one speaker each, for three minutes. The floor can then decide to vote, or open a rounds of debate with self-appointed speakers.

Current events (/agenda política/) are party leaders' position-taking venue. The Junta determines up to two themes for debate before consideration of reports and bills, party leaders appointing one speaker each. The promoting party speaker gets first recognition for 10 minutes, others 5 minutes each, and talk in reverse-size order. Current events debate cannot exceed two hours per session. 

  \subsection{Recognition-granting motions}
Debate under such rules becomes a succession of punctuated, mostly uninterrupted short speeches. Members can approximate back-and-forth talk, at least occasionally, by catching the president's eye from their seats in order to interrupt with a motion. The president has discretion to deny, or grant up to three minutes to elaborate. Such motions are distinct from points of order (which members can also make, see Reglamento art. 114 for typified motions). They grant recognition to speak. One (/cuestionamiento al orador/) to interrogate the speaker, who must also accept the question be made. Another is (/alusiones personales/), to give right of reply to alluded members by recognizing them right after the speaker ends. And (/rectificación de hechos/) wind up an additional name at the end of the list of speakers. 

  \subsection{Party discipline as alternative to centralized agenda power}
The Cámara's debate rules are ill-designed to prevent plenary bottlenecks \citep{cox.2006}. Even in the presence of a majority party, individual members retain speaking rights that water down attempts by the Junta to cartelize the legislative process. So how does the Cámara prevent dilatory motions to get things done? The answer is parties. Party discipline operates as an alternative to agenda cartelization in many systems \citep{prata.2006}. 

Cohesion is near perfect across parties. \citet{tellez-del-rio.2018} computed frequencies with which deputies voted against a majority of their party (excluding unanimous votes). The mean he reports for the 1997--2018 period stands at 2 percent, 3.4 percent when abstentions are coded as votes against the party majority (p. 25). 

Three former deputies from the larger parties offered quick impressions on internal party speech rules upon request.\footnote{Email exchanges with Fernando Rodríguez Doval (PAN), Lupita Vargas Vargas (PRI), and an anonymous former lawmaker from the left, June 17th, 2020.} One commonality (at least in this very small sample) is the informal erosion of formal individual members' debate rights in favor of centralized speech allocation \citep[cf.][]{cox.1987}. The PAN relies on a debate whip (subcoordinador de debate parlamentario) in charge of selecting speakers in debates. When two members wish to speak at once, the whip would let them figure who would get the party's slot in the debate, who would then speak for or against. The PRI leadership sets apart issues of party interest, appointing every speaker when debated. Members would communicate their wish to speak on unwhipped issues to their state caucus leader, who would seek authorization with party whips. Rules give parties one speaking slot each in many debates, regardless of size. Distributive conflict over speech is therefore more acute for larger parties, with longer speaker lists. A must for a dissenting member is a solid understanding of the Rules. That member can thus make individual speaking rights effective by introducing suspensive motions or amendments, both of which come equipped with recognition to take the floor. 

Party leaders move the strings of lawmaking. Their influence, however, derives almost exclusively from party discipline (near-perfect across the board) and not from agenda power (which is quite diffuse). 

\section{The determinants of legislative debate} % [ca 2500]
 
  %% In this empirical section, we want you to explore how intra and interparty, as well as individual features, play a role in determining the likelihood that MPs take the floor (and how often they will take the floor). 
%% The section is divided into subsections. The first uses descriptive statistics and bivariate analysis. The second turns to multivariate analysis. 
%% Please note that the contents of this section are rigid. Please save important aspects that are relevant to your case to a subsequent section on ‘Country Specific Matters’. 

  \subsection{Data and methods}

Digitized speeches come from the stenographic service (scraped from http://cronica.diputados.gob.mx/). I relied on regular expressions to de-htmlize the text and identify speakers and their speech, turning text into data for analysis.\footnote{Data analysis was performed in R \citep{r.cite}, all code is available at https://github.com/emagar/legdeb. I relied on libraries lme4 \citep{r.lme4}, lubridate \citep{r.lubridate}, margins \citep{r.margins}, MASS \citep{r.mass}, plyr \citep{r.plyr}, stargazer \citep{r.stargazer}, and zoo \citep{r.zoo}.}

The dependent variable is a member's participation in plenary debate during legislative periods (see the appendix for terminology). The 60th, 62nd, and 64th Legislatures had six, eight, and five periods, respectively, totaling nineteen. Three are extraordinary periods, the rest ordinary. Mean days per period was 6.7 for the former, 31.4 for the latter, so the debate models control for period length. 

I use two specifications of the dependent variable. One is *speeches(i,p)* equal the number of days that member i took the floor in period p. Owing to the permissive agenda, days when a deputy spoke from her seat by means of motions, without taking the lectern, count as debate. Days when deputy i spoke fewer than 50 words in total are arbitrarily considered non-debate and dropped, adding zero towards the member's aggregates. Since officers do not participate in legislative debate, all steering speech, as when the president recognizes a deputy or the secretary calls a voice vote to dispense reading of the bill, was also removed. So was speech by non-deputies, as in cabinet member hearings. Everything remaining is considered debate, members' daily totals added across sessions in the same period to produce aggregates for analysis.

The other specification is *words(i,p)* equal the number of words that member i spoke in period p divided by the number of days that i served as a proportion of all session days in period p---members can take leaves of absence, so many served less than the full period. So the denominators for two members i and j who both spoke 2 thousand words, i served uninterrupted throughout period p, j served only half of period p, are 1 and 0.5, respectively. This makes words(i,p)=2000 but words(j,p)=2000/0.5=4000 instead. 

%As in other chapters, the dependent variable is the number of words that members spoke in the chamber. A given diputado's words throughout a plenary session were summed into a daily total. Daily totals less than 50 words were arbitrarily interpreted as not constituting speech and removed from the data (ie., the member received a value of zero words that day). Thus filtered, members' daily totals were added across sessions in the same period, producing word aggregates for analysis.

Table \ref{T:descriptives} has a summary of the dependent variable along others of interest. Member-period observations total 9494. The median member spoke once per period, delivering 607 words relative to days in office (593 words per period in absolute terms). At nearly 1400 words per period, means are substantially higher owing to a right-skewed speech distribution portrayed in Figure \ref{F:dv-hist}. Relevant to the choice of estimation methods, speech data might are not evidently over-dispersed (at 3.1, the standard deviation is not that much higher than the mean of 2.1), so both negative binomial and poisson regression will be used for estimation. And the nearly two out of five members who uttered not a single word in the period (37.6 percent) suggest adoption of a zero-inflated approach.

\begin{table}
  \begin{scriptsize}
    \begin{verbatim}
Part A: Continuous variables
|                              |   min | median |  mean |     sd |   max |    N |
|------------------------------+-------+--------+-------+--------+-------|------|
| N speeches (DV1)             |     0 |      1 |   2.1 |    3.1 |    37 | 9494 |
| N words / exposure (DV2)     |     0 |    607 |  1391 | 2716.3 | 50291 | 9494 |
| N words                      |     0 |    593 |  1366 | 2682.3 | 50291 | 9494 |
| Days in office (exposure)    |     1 |     30 |  26.7 |   11.2 |    40 | 9494 |
| Party share                  |   0.4 |     25 |  29.2 |   15.9 |    51 | 9494 |
| Seniority                    |     0 |      1 |   1.7 |    2.2 |    17 | 9494 |
| Previous terms               |     0 |      0 |   0.3 |    0.6 |     4 | 9494 |
| Age                          |    21 |     46 |  45.9 |   10.1 |    78 | 7332 |

Part B: Dichotomous variables
|          |    0 |    1 | tot |    N |
|----------+------+------+-----+------|
| Spoke    | 37.6 | 62.4 | 100 | 9494 |
| Majority | 86.6 | 13.4 | 100 | 9494 |
| Leader   | 98.3 |  1.7 | 100 | 9494 |
| Chair    | 90.6 |  9.4 | 100 | 9494 |
| SMD      | 39.3 | 60.7 | 100 | 9494 |
| Suplente | 94.2 |  5.8 | 100 | 9494 |
| Extraord | 84.5 | 15.5 | 100 | 9494 |
| Female   | 64.2 | 35.8 | 100 | 9494 |
| 60th     | 68.2 | 31.8 | 100 | 9494 |
| 62nd     | 57.6 | 42.4 | 100 | 9494 |
| 64th     | 74.2 | 25.8 | 100 | 9494 |
| PAN      | 72.8 | 27.2 | 100 | 9494 |
| PRI      | 72.8 | 27.2 | 100 | 9494 |
| Left     | 70.0 | 30.0 | 100 | 9494 |
    \end{verbatim}
  \end{scriptsize}
\caption{Variable descriptives}\label{T:descriptives}
\end{table}

\begin{figure}
  \centering
  \begin{tabular}{cc}
    \includegraphics[width=.49\columnwidth]{../plots/dv-histogram.pdf} &
    \includegraphics[width=.49\columnwidth]{../plots/dv-nspeech-histogram.pdf}
  \end{tabular}
    \caption{The dependent variable, number of words (left) and number of speeches (right). The column under a star in the left panel is fictitious, reporting 217 member-periods with 8 thousand words or more (2.2 percent of all, the actual distribution spreads these observations, with increasing sparseness, from 8000 to 50291).}\label{F:dv-hist}
\end{figure}

%% \begin{figure}
%%   \centering
%%     \includegraphics[width=.8\columnwidth]{../plots/nspeakers.pdf}
%%     \caption{Number of daily speakers}\label{F:nspeakers}
%% \end{figure}

Debate length is easier to grasp when expressed as daily totals instead of the period totals analyzed. In the median session, 36.5 different speakers contributed to daily debate, and six days had over 100 speakers. Figure \ref{F:quantiles} portrays member daily aggregates across the periods analyzed. For clarity, this plot includes speakers only (keep in mind that non-speakers are included in the period aggregates analyzed below.) Solid points report median daily speech length in words. With few exceptions, period medians are much the same as the overall median daily speech length of 599 words. Mild term effects show up too, the 60th medians slightly above and the 64th slightly below the overall median. Horizontal lines report the spread of the central portion of the density---the thicker line is the inter-quartile range, the thinner connects the first and ninth deciles. Period distributions are, in general, similar. The clearest exceptions are extraordinary periods, drawn in gray. The models therefore include controls for term and ordinary session effects.

\begin{figure}
  \centering
    \includegraphics[width=.5\columnwidth]{../plots/quantiles-periodo.pdf}
    \caption{Daily speech length by legislative period observed. The plot excludes non-speaking members. Solid points indicate the median speech length in the period. Thick and thin lines connect the 25--75 and 10--90 percentiles, respectively. Hollow points are minima and maxima. Ordinary periods in black, extraordinary periods in gray.}\label{F:quantiles}
\end{figure}

Hollow points are minima and maxima. Diputada Valentina Batres holds the record for delivering the longest speech in the three terms examined. At 15,932 words, her speech delivered March 11th, 2008 is 50 percent longer than the runner-up and has about as many words as \emph{Don Quijote de la Mancha}'s chapters 1 through 7 (forty-five pages in the edition I own). Batres and legislators close to AMLO used dilatory tactics throughout that day's session, delaying the vote of a national geostatistics law. I suspect that filibustering was probably aimed at a bill down the line, with plainer distributive effects \citep[cf.][]{wawro.schickler.filibuster.2007}. A systematic study of filibustering in the Cámara is worthy of further study. The names associated with extreme member-periods (those grouped in the left panel of Figure \ref{F:dv-hist}'s starred column) are few: only nine deputies repeatedly surpassed 20 thousand words per period, mostly in the 62nd term. They are routine filibusters.
%Most are in AMLO's present government. 

%Sistema Nacional de Información Estadística y Geográfica
%Deputies close to AMLO were adoting dilatory tactics. Batres requested the addition of a point to JuCoPo's order of the day. Mesa Directiva denied, so a PRD faction threatened to take over the Tribuna (?). Batres introduced motion to suspend and other dilatory tactics (ley sis nal informacion inegi), then filibustered (called art. 103 ley reglamentaria, granting her 30 minutes to present minority vote despite JuCoPo's day aggreement to limit to 10 minutes.)

% Chapters 1 through 6 of El Quijote 13,049, 40 pages in my edition.
% Chapters 1 through 7 of El Quijote 14,916, 45 pages in my edition.
% Sonnets and Chapters 1 through 7 of El Quijote 16,191, 45+ pages in my edition.
% Chapters 1 through 7 of El Quijote 17,912, 45+ pages in my edition.

%The log scale magnifies the low side of the distribution. But it also blurs subtle but revealing differences in the high side. From 60th to 62nd, max went up while percentiles 75 and 90 remain at similar levels. Like Batres, an unusually high top decile consists of attempts to disrupt debate through filibustering. Dilatory tactics went down in 62nd compared with 60th, and substantially so in the 64th with a single party majority.


%% \singlespacing
%% \begin{footnotesize}
%% \begin{verbatim}
%% Words per day
%% | Legislatura    | min | 10% | 25% | 50% | 75% |  90% |   max |
%% | 60th           |  50 | 137 | 392 | 652 | 901 | 1215 | 15932 |
%% | 62nd           |  50 | 193 | 438 | 611 | 850 | 1254 |  9765 |
%% | 64th (partial) |  50 | 142 | 327 | 547 | 730 |  975 |  6358 |
%% \end{verbatim}
%% \end{footnotesize}
%% \doublespacing


  \subsection{Gender and seniority}

%% Please begin this subsection with a descriptive statistics table, including both DVs and all independent variables. Please include Mean, Std. Dv., Min, Max

%% Bivariate Analysis: Here, we cover two individual aspects. For each of these aspects, please produce a figure and discuss it. The editors will be sending you a do-file to harmonize the aspect of the figures across the volume. The following features should be discussed here:

%% 1. Gender: what is the impact of gender in floor access? Do women have access to the floor that is commensurate to their numerical presence in the legislative party? Operationalize gender as a dummy variable: Women takes the value of 1, and Men 0. Please use this coding also in the multivariate analysis (several drafts coded male MPs with 1 and female MPs with zero).

  The relationships of gender and seniority with floor access are of interest across chapters. Of 1710 members observed, 39 percent are women (see Table \ref{T:women}). Owing to stricter quotas, 47 percent of the 64th Legislature were women, up from 28 in the 60th \citep{piscopo.2016}. Women participation in debate exceeds their numerical presence: despite subrepresentation among committee chairs and party leaders (but not Cámara presidents), 41 percent of both speeches and total words were delivered by women in the floor. A degree of concentration is also manifest, as women represented 37 percent of unique speechmakers, who  spoke more often and quite longer. 

\begin{table}
  \begin{scriptsize}
    \begin{verbatim}
|                   | % women |    of |
|-------------------+---------+-------|
| Members           |      39 |  1710^|
| -60th             |      28 |   603 |
| -62nd             |      41 |   640 |
| -64th             |      47 |   531 |
| Cámara presidents |      35 |    31 |
| Committee chairs  |      25 |   143 |
| Party leaders     |      21 |    24 |
| - major party     |       0 |    12 |
| - opportunistic   |      42 |    12 |
| Speechmakers      |      37 |  5926 |
| Speeches          |      41 | 23601 |
| Words spoken      |      41 | 17.5M |
|-------------------+---------+-------|
|^Returning members counted once only.|
|-------------------+---------+-------|
    \end{verbatim}
  \end{scriptsize}
\caption{Women representation and debate}\label{T:women}
\end{table}

%% 2. Seniority: what is the impact of career stage in the likelihood of accessing the floor? Do more senior legislators get more access to the floor because they have offered party leaders signals of their work as party agents? Operationalize Seniority as a continuous variable that measures the number of years the MP has spent in the legislature.

Single-term limits offer little leverage to evaluate how seniority impacts floor access. Members wishing to return had to wait one term at least. It is remarkable that, despite this, 14 percent of members had previous federal deputy experience. This hints that the removal of single-terms will not be irrelevant due to lack of static ambition (as in Argentina, for instance). Freshmen spoke 1181 words per period on average, compared to 2082 for members with past terms as deputies. Member-periods with one past term made 37 percent more speeches and spoke 65 percent more words than those with none; with two past terms instead of one, 35 percent more speeches and 49 percent more words; but those with more than two past terms gave 29 percent less speeches and 52 percent fewer words than those with two. This drop could be attributable to earlier recruitment of senior members, antedating competitive politics; or it could be due to higher likelihood that senior members occupy positions that might depress willingness to speak despite floor access possibilities. The multivariate analysis might shed some light.  

\begin{table}
  \begin{scriptsize}
    \begin{verbatim}
| Past  | Mean number | Mean number | Member- |
| terms | of speeches |    of words | periods |
|-------+-------------+-------------+---------|
| 0     |         1.9 |        1181 |    7550 |
| > 0   |         2.8 |        2082 |    1944 |
| 1     |         2.6 |        1944 |    1531 |
| 2     |         3.5 |        2892 |     330 |
| > 2   |         2.5 |        1413 |      83 |
|-------+-------------+-------------+---------|
    \end{verbatim}
  \end{scriptsize}
\caption{Seniority and floor access, member-periods}
\end{table}
  
  \subsection{A model of debate}

%% In this subsection, we run multivariate analysis explaining the determinants of participation in legislative debates. Before we go into the details of the analysis, let’s pause to discuss the unit of observation. As a rule of thumb, the unit of observation (each row in the data matrix) should be the MP over legislative term. However, some of you expressed the desire to have more disaggregated levels of analysis. For example, to observe the MP per day, MP per month, MP per legislative session. We are happy with the choice that you make, provided that you explain it in a clear way to help the readership making sense of it. 

%% In this section, we want you to run models using two different dependent variables:

%% 1. Number of speeches that a legislator delivered in the time unit you defined (presumably, for most of you, the legislative term). In doing so, use a negative binomial regression.
%% 2. Number of words divided by exposure (see below how to operationalize) that a legislator delivered in the time unit you defined (presumably, for the most of you, the legislative term). In doing so, use an OLS.
%% 3. For both cases, please included fixed-effects for the time period of interest (e.g., for the legislative term or the legislative session, depending on your choice). 
%% 4. Please include standard errors clustered at the legislator level.

%% What are the independent variables that you should use? As we talked over the workshop, in this section we need you to make models that are the same (or at least very similar) for all countries. In the country-specific section, you are free to make model extensions to account for country specification

%% In this section, please include the following covariates:

%% 1. Gender – dummy variable that takes a value of 1 for Women and 0 for Men
%% 2. Party Size – continuous variable that measures the absolute number of members of the legislative party
%% 3. Seniority – continuous variable that measures the number of years the legislator has been in the parliament
%% 4. Age
%% 5. Age Squared
%% 6. Party Family (Dummy variables, using one of party families as reference category)
%% 7. Committee Chair – dummy variable that takes a value of 1 if the MP holds a committee chair and 0 for all others
%% 8. Minister – dummy variable that takes a value of 1 if the MP is a minister and 0 otherwise
%% 9. Government party member – dummy variable that takes a value of 1 if the MP belongs to a legislative party that belongs the government and 0 otherwise. Note that we only consider parties that are formally in a coalition (i.e., have members in the executive). Supporting parties, e.g. contract parliamentarism, do not count towards government parties.
%% 10. Legislative Party Leadership – dummy variable that takes a value of 1 if the MP belongs to the leadership of the parliamentary party group
%% 11. Party Leader – dummy variable that takes a value of 1 if the MP is the party leader and 0 otherwise
%% 12. Exposure (logged) – continuous variable that measures the percentage of time in which the MP held to her seat in parliament during the unit of time defined in your chapter. For example, if you are using a MP-legislative term unit of observation, in this variable you need to include the percentage of time during the legislative term in which the MP was in the parliament. If MP was in parliament for whole session that would be 1. If the MP joined the parliament later, it could be .7 or .8. If you are using month as the time unit, the same rationale applies. The logged version should *only* be included in the count models (negative binomial). 

%% How to build the DV for the OLS models: 

%% Where the outcome is the number of Words, you should use Exposure as the denominator to create a ratio. The said ratio should consist of the "total number of words legislator i delivered during legislative term t/percentage of time legislator i sat in legislative term t”.

%% The rationale behind this measure is that we want to capture the time that each legislator sits in parliament during a given session. Obviously, a legislator who sits for the duration of the terms has higher chances of taking the floor than a legislator that takes her sit in the middle of the term.

%% Don’t forget to include Term FE, plus clustered standard errors at the MP level.

%% Please produce a table including both the negative binomial models and the OLS. For negative binomial models, please report the AIC.

%% Please include up to 5 models in the tables. Consider using a step-wise approach to regression by including covariates into the equation that make most sense in your context. 
%% Ultimately, we need 2 final models, where all variables are included – one where the dependent variable is the Number of Speeches and the other where the dependent variable is the Number of Words.

%% As a default we consider the following variables as explanatory:
%% 1. Gender
%% 2. Seniority
%% 3. Committee Chairs
%% 4. Minister
%% 5. Government party member
%% 6. Legislative Party Leadership
%% 7. Party Leader

%% The following variables are considered controls:

%% 1. Age
%% 2. Age Squared
%% 3. Party Family
%% 4. Exposure (logged)

%% Please feel free to use variables interchangeably between the two categories depending on the context. 

%% Please plot marginal effects using the full specification of the negative binomial model. In the said plot, please include explanatory variables only. Controls variables can be omitted.


To analyze participation in floor debates, I fit multivariate event count models to words spoken. In the right side are status variables, member characteristics, and controls. Units are member-periods. 

    \subsubsection{Status variables}

A dummy for *majority* status indicates members from Morena in the 64th Legislature---the only party controlling over 50 percent of seats. If debate is an (imperfect) substitute for legislative outcomes, then minority members demand more frequent floor participation \citep{proksch-slapin2015book}. On the contrary, if members put value on debate per se, the majority may demand it as much as others, possibly with better access to the floor. Next, a dummy for committee *chair* status. When producing a report, the chair has privileged access to the floor, and this should translate into more speech. A dummy for party *leader* status completes this set. Leaders allocate party speakers. Whether or not they take advantage of this privilege remains an open question, a good leader ought to distribute the goodies, or risk removal. 

    \subsubsection{Member variables}

Aside from *woman* and *seniority*, regressors in this group include *smd*, a dummy equal one for members elected in single-member districts. Systematic differences in members' pork requests are attributable to the method of election \citep{kerevelPork2015}, which may also translate into higher demand for access to the floor. I also interact this regressor with a dummy indicating the 64th Legislature, which dropped single-term limits (*smd x reelection*). The more personal vote should generate higher demand for floor access. *Party size* is the percentage of seats the member's party holds. Larger parties must divide the slot that all parties get to take the floor among more members, and this should show up as a negative regression coefficient. And a dummy *suplente* controls for substitute members. Regressors not in the right side include members' ages due to incomplete data, and party ideology, which made no difference in the estimates.

    \subsubsection{Other controls}

Also in the right side are dummies for the *62nd* and *64th* terms (the 60th is the baseline) and another for *extraordinary* periods. Finally, with the option to take leaves of absence and have suplentes take over, some members served incomplete periods. The *exposure* is the number of days that the members served in the period, logged. Higher exposure offers more opportunities for floor access. 

%A pair of dummies controls for the party's ideology. *Right* equals 1 for members of the right-of-center PAN, 0 otherwise. Left equals 1 for PRD members in 60th and 62nd, and Morena member in the 64th, 0 otherwise. The omitted group includes members of the PRI and the smaller opportunistic parties. The dummy should capture any systematic effect of left's more frequent filibustering attempts. (There is no a priori expectation associated with left and right.)

Table \ref{T:regs} reports the estimation of six different model specifications. In the left side are both flavors of the dependent variable. Models of words relative to tenure were fit with ordinary least squares (1, 2, and 3), models of the number of speeches with negative binomial regression (4 and 5) and zero-inflated poisson regression (6). Specifications vary the regressors. Models 2, 3, 5, and 6 include fixed term effects, capturing any heterogeneity between Legislatures that are pooled together. Model 3 estimates separate error terms for each member, intended to capture individual heterogeneity. And model 6 accounts for the excess of zeroes in the distribution seen in Figure \ref{F:dv-hist}. The overall fit is correct across models, likelihood ratio tests (not reported) reject the intercept-only model with much confidence. 

\begin{table} \centering 
  \begin{tiny}
    \begin{verbatim}
==================================================================================================================================
                                     DV = Words/exposure in period                              DV = Speeches in period               
                      --------------------------------------------------------------   -------------------------------------------
                              (1)                       (2)                 (3)            (4)            (5)           (6)     
----------------------------------------------------------------------------------------------------------------------------------
Exposure (logged)                                                                       0.96***        1.30***        1.06***   
                                                                                         (0.02)         (0.05)        (0.04)    
                                                                                                                                
Majority                  1,032.85***               1,494.59***          848.60***      1.22***        0.98***        1.11***   
                            (91.01)                  (137.23)            (226.06)        (0.05)         (0.06)        (0.04)    
                                                                                                                                
Party leader              2,121.40***               1,906.26***         1,292.59***     0.34***        -0.04***      -0.04***   
                            (206.38)                 (205.16)            (310.20)        (0.08)        (0.001)        (0.001)   
                                                                                                                                
Comm. chair                239.92***                  145.14*              51.54        0.27***        0.31***        0.22***   
                            (87.49)                   (86.86)            (146.09)        (0.04)         (0.08)        (0.04)    
                                                                                                                                
Seniority                  224.72***                 258.78***           262.58***      0.11***        0.25***        0.11***   
                            (48.14)                   (47.53)             (85.76)        (0.02)         (0.04)        (0.02)    
                                                                                                                                
Woman                      170.47***                 131.49**              19.91        0.14***         -0.06*         -0.02    
                            (54.90)                   (54.89)             (99.70)        (0.03)         (0.03)        (0.02)    
                                                                                                                                
Party size                 -67.47***                 -72.05***           -63.23***      -0.05***       0.25***        0.10***   
                             (2.00)                   (2.26)              (3.85)        (0.001)         (0.05)        (0.03)    
                                                                                                                                
SMD                          -25.24                   -91.94              -115.00         0.03         0.12***        0.10***   
                            (55.84)                   (64.91)            (115.48)        (0.03)         (0.02)        (0.01)    
                                                                                                                                
SMD x reelect                                        267.78**              7.68                        0.09***        0.04**    
                                                     (120.85)            (189.51)                       (0.03)        (0.02)    
                                                                                                                                
Suplente                   -297.84***               -366.14***           -349.00**      -0.19***        -0.10*       -0.24***   
                            (110.56)                 (108.95)            (140.66)        (0.06)         (0.06)        (0.05)    
                                                                                                                                
62nd Leg.                                            698.07***           836.08***                     0.25***        0.24***   
                                                      (62.82)             (96.43)                       (0.03)        (0.02)    
                                                                                                                                
64th Leg.                                             -114.31            508.02***                     0.19***        0.17***   
                                                     (114.68)            (165.84)                       (0.05)        (0.03)    
                                                                                                                                
Extraordinary                                      -1,102.30***        -1,109.17***                    0.65***        0.68***   
                                                      (73.17)             (48.30)                       (0.08)        (0.07)    
                                                                                                                                
Constant                  3,077.93***               3,075.94***         2,807.58***     -1.53***       -2.91***      -1.91***   
                            (71.07)                   (86.37)            (144.17)        (0.08)         (0.16)        (0.13)    
                                                                                                                                
----------------------------------------------------------------------------------------------------------------------------------
Fixed effects                  no                      term                term            no           term           term
Random effects                 no                       no                member           no            no             no
Estimation method             OLS                      OLS                linear         negative      negative    zero-inflated
                                                                       mixed-effects     binomial      binomial       poisson   
----------------------------------------------------------------------------------------------------------------------------------
Observations                 9,494                     9,494               9,494         9,494          9,494          9,494    
R2                            0.15                     0.18                                                                     
Adjusted R2                   0.15                     0.18                                                                     
Log Likelihood                                                          -85,188.83     -16,232.18     -16,126.53    -17,305.06 
theta                                                                                1.55*** (0.05) 1.65*** (0.05)             
Akaike Inf. Crit.                                                       170,407.70     32,484.36      32,281.06                
Bayesian Inf. Crit.                                                     170,515.00                                              
Residual Std. Error   2,504.36 (df = 9485)     2,465.37 (df = 9481)                                                             
F Statistic         210.29*** (df = 8; 9485) 170.19*** (df = 12; 9481)                                                          
==================================================================================================================================
Note:                                                                                                  *p<0.1; **p<0.05; ***p<0.01
\end{verbatim}
  \end{tiny}
  \caption{Models of legislative debate (standard errors in parentheses)} 
  \label{T:regs} 
\end{table} 

Interesting patterns emerge from coefficient estimates. Party size exerted a negative and statistically significant effect in member floor access across specifications. This is easier to interpret from OLS coefficients: other variables constant, changing the party size from large (40 percent of seats) to small (15 percent) associates with a predicted drop of 1,700 words by member in the period. Martin Luther King took 16 minutes to deliver his famous "I have a dream" speech, which approximates that word count. I also find a positive, significant, and large effect of majority status, which acts against size. Far from letting legislative accomplishments speak for themselves, majority members take the floor systematically more than those of similar-sized parties. Figure \ref{F:predict} demonstrates the discontinuity through simulation with model 5 parameters. As party size crosses the majority threshold the member gets a bonus, delivering a number of speeches comparable to a party with 25 percent of seats.

%Note that the relative equals absolute words divided by the share of period p's duration that member i served, so the *exposure* is in the left side of OLS models and not in the right. 
%*words(i,p) x days(p)  / exposure(i,p)*

\begin{figure}
  \centering
    \includegraphics[width=.67\columnwidth]{../plots/predictedWords.pdf}
    \caption{Predicted number of speeches by party size. Lines report point predictions using model 5, bands are 95-percent confidence intervals. Miniature gray points are observed members' party sizes, x- and y-jittered for visibility.}\label{F:predict}
\end{figure}

Other forms of status also associate positively to floor access, but results are sensitive to model specification. Party leadership exerts a substantially larger effect than majority status on speech length, but much smaller on the number of speeches. Leaders get privileged floor access and appear to specialize in longer speeches, probably on more important legislation. Committee chairs also deliver more speeches than other members, but controlling for term and member effects bears upon OLS coefficient significance, both substantially and statistically, hinting to important differences in speeches length across committee jurisdictions and individuals.

%Figure \ref{F:avgmgeff} reports changes in the average predicted number of words associated with unit changes in explanatory variables, using model 5 estimates. Marginal effects make negative binomial regression coefficients interpretable. It is clear in the plot that the largest effect by far is attributable to majority status. Other things constant (at mean values), members in the majority party each gave between 1.75 and 2.25 more speeches per period than the rest of the Cámara. Multiplication of that average by Morena's 254 members in the 64th Legislature produces over 500 additional plenary speeches---47 percent of all words in the median ordinary period. OJO: NEED TO DEDUCT SIZE 51 --- better with predicted plot

I also find positive effects of seniority and gender that resonate with the bivariate patterns of floor access. The coefficient for *women* is not robust to random member effects nor to accounting for zero-inflation. This is probably due to the concentration of debate by women highlighted above, some deputies taking the floor disproportionally more than others. Overall, the effect of gender appears to be on par with that of one additional term of seniority. 

\begin{figure}
  \centering
    \includegraphics[width=.67\columnwidth]{../plots/avgMgEffects.pdf}
    \caption{Average marginal effects from model 5. Circles report the effect in the expected number of speeches per period of a unit change in each independent variable, all else at mean values; bars are 95-percent confidence intervals.}\label{F:avgmgeff}
\end{figure}

%% model 5 (fit2 in code) average marginal effects
%%      factor       AME       SE        z      p     lower     upper
%%  ev.pot.dys   51.3100   2.6546  19.3285 0.0000   46.1070   56.5129
%%        dmaj 1464.4438 211.4850   6.9246 0.0000 1049.9408 1878.9469
%%    size.maj  -72.7583   4.9124 -14.8113 0.0000  -82.3863  -63.1303
%%     dleader  460.6492 298.1915   1.5448 0.1224 -123.7954 1045.0937
%%      dchair  451.9050 130.7671   3.4558 0.0005  195.6061  708.2039
%%        dsmd -135.7903  93.9759  -1.4449 0.1485 -319.9796   48.3991
%%      dsmd64  359.7821 175.2500   2.0530 0.0401   16.2985  703.2657
%%   seniority  241.5023  69.7417   3.4628 0.0005  104.8111  378.1935
%%        dfem   93.4748  79.3548   1.1779 0.2388  -62.0578  249.0074
%%        dsup -507.2173 160.3330  -3.1635 0.0016 -821.4642 -192.9703
%%         d62  259.0260  89.6724   2.8886 0.0039   83.2712  434.7807
%%         d64 -132.3693 165.2884  -0.8008 0.4232 -456.3285  191.5899

A null finding of interest involves the method of election. The coefficient for *smd* is indistinguishable from zero across models. Figure \ref{F:avgmgeff} reports average marginal effects to interpret negative binomial regression coefficients: in contrast to PR members and with all other regressors at their mean, deputies elected in SMDs spoke slightly less, about 125 words in the period; the 95-percent confidence interval barely excludes the zero and this signal can't be discarded as product of chance alone. But look at the change in slope when interacted with reelection: this marginal effect is not just positive, but sufficient to cancel the negative pull of SMDs. Now a signal is discernible from random noise, even after controlling for majority status (the other big change in the 64th Legislature). Figure \ref{F:predict} makes this effect plain, a gap separates confidence intervals of predicted speeches by SMD members who can reelect and the term-limited. This finding hints to the invigoration of the personal vote after the removal of term limits and is worthy of more careful examination. 

\section{Discussion: minority rights} % [ca 1000 words]

%% In this section, you can feel free to make model extensions that have interest in the light of the chapter you are exploring. Please do not forget to explain the variables in use, as well as why they are important for your country. Include a table of results plus a plot for marginal effects. 

(Forthcoming.)

Tension lies at the heart of legislative debate in the Cámara. On one hand, intra-party institutions have informally, but effectively managed to reign in members' capacity to take the floor. The effects that multivariate models uncovered for the majority, for leaders, and for committee chairs are all channeled through party structures in the Junta. On the other hand, formal institutions grant individual members formal rights of recognition to take the floor and, we have seen, these take many guises. The effect attributable to SMDs after the removal of term limits is, in all likelihood, associated to renovated personal vote incentives that members face. 

Whether or not the informal solution to avoid plenary bottlenecks will continue to operate as it has so far is uncertain. Incumbents, some of them at least, may soon start overwhelming the system in their need to strengthen their electoral connection. The collapse of the three-party system in 2018 also plays against. Perhaps the heterogeneous coalition that gave Morena unified control of government will manage to consolidate, imposing a new informal arrangement, in spite of the 2020 covid depression.

In any event, examination of legislative debate has offered an interesting and illuminating perspective on some of the challenges that Mexican parties now face.   

\section{Conclusion} % [ca 500 words]

(Forthcoming)

%concluding discussion of general patterns of speechmaking (institutions and empirical results in terms of background variables)



%% suspension of rules typified only for discharge, two-thirds

%% The constitution sets the quorum at half chamber membership.

%% Reglamento amendments by 2/3 vote

%% Suspension of rules by Conferencia always a choice, but only typified for committee of the whole. Art 77 cpeum. Risks toma de tribuna.

%% Presiding officer can summon police to restore order. 
%% Can summon public force, but in practice never used. 
%% Can kill the mike, but others can raise their voices

%% 4. Para atender una situación no prevista en el Reglamento, el Presidente podrá dictar una resolución de carácter general, siempre que haya la opinión favorable de la Mesa Directiva y de la Junta. En caso contrario, este tipo de resoluciones sólo tendrán efecto con la aprobación de la mayoría simple del Pleno.



%% - A Legislature (with Roman numerals for reasons I ignore) is an elected chamber for a legislative term, called a Congress in the U.S. Concurrent with presidential elections the chamber of deputies renovates in whole, and again at the presidential mid-term. Diputados remain three years in office and were single term-limited up to 2021. The 2021 mid-term election will be the first since 1932 to allow incumbents on the ballot, a major change in Mexican legislative politics.
%% - Legislative years break into two "ordinary periods", one covering the months of September through December, inclusive, another February through April, also inclusive. "Extraordinary periods" may be convened during the recess in order to consider a specific bill. Analysis aggregates each member's speeches in the duration of a given period (merging together all extraordinary periods that year, if any). So members in a legislative year like 2012-13 (that had no extraordinary periods) have two word aggregates in the dataset, one for each ordinary period; in a year like 2013-14 (that did), they have three word aggregates in the data. Periods are the units of aggregation in the analysis. 
%% - A plenary session is a specific date in the calendar when diputados met. During ordinary periods, sessions are usually held on Tuesdays and Thursdays, and may be scheduled in other weekdays if the Jucopo so decides. Diputados met on forty and thirty-one days in the first and second ordinary periods of 2013-14, respectively, and nine days in extraordinary periods, for a yearly total of eighty session days. (A session in North-American legislative parlance is a Mexican period.)

\singlespacing

\section{Appendix: terminology}

%\subsection{Terminology}

- A *Legislature* is an elected chamber for a legislative term, between two congressional elections. The Mexican Congress relies on Roman numerals to distinguish consecutive Legislatures since the second half of the Nineteenth century.

- Legislative years break into two *ordinary legislative periods*, one covering the months of September through December, another February through April, all inclusive. *Extraordinary legislative periods* may be convened during the recess in order to consider a specific bill. Analysis aggregates each member's speeches in the duration of a given period (merging together all extraordinary periods that year, if any). So members in a legislative year like 2012-13 (that had no extraordinary periods) have two word aggregates in the dataset, one for each ordinary period; in a year like 2013-14 (that did), they have three word aggregates in the data. Periods are the units of observation in the analysis. 

- A *plenary session* is a specific date in the calendar when diputados met. During ordinary periods, sessions are usually held on Tuesdays and Thursdays, and may be scheduled in other weekdays if the Junta so decides. Diputados met on forty and thirty-one days in the first and second ordinary periods of 2013-14, respectively, and nine days in extraordinary periods, for a yearly total of eighty session days.

%% \subsection{Alternative model specifications}

%% Forthcoming. 

%% \begin{table} \centering 
%%   \begin{tiny}
%%     \begin{verbatim}
%% ==================================================================================================================================
%%                                     DV = Words/exposure in period                                  DV = Words in period               
%%                       --------------------------------------------------------------   -------------------------------------------
%%                               (1)                       (2)                 (3)            (4)             (5)            (6)     
%% ----------------------------------------------------------------------------------------------------------------------------------
%% Exposure (logged)                                                                        0.93***         0.95***        0.34***   
%%                                                                                          (0.04)          (0.04)         (0.02)    
                                                                                                                                  
%% Majority                  1,029.48***               1,489.83***          835.24***       1.03***         1.04***        0.78***   
%%                             (91.07)                  (137.37)            (226.39)        (0.09)          (0.14)         (0.06)    
                                                                                                                                  
%% Party leader              2,186.15***               1,972.75***         1,309.31***       0.40*           0.33          0.28***   
%%                             (206.55)                 (205.36)            (310.46)        (0.21)          (0.21)         (0.07)    
                                                                                                                                  
%% Comm. chair                247.65***                  139.20               88.01         0.37***         0.32***        0.14***   
%%                             (89.79)                   (89.29)            (152.24)        (0.09)          (0.09)         (0.04)    
                                                                                                                                  
%% Seniority                  145.30***                 180.04***           203.98**        0.16***         0.17***        0.12***   
%%                             (48.19)                   (47.59)             (85.93)        (0.05)          (0.05)         (0.02)    
                                                                                                                                  
%% Woman                      164.02***                 125.84**              17.03          0.08            0.07           0.03     
%%                             (54.92)                   (54.94)             (99.85)        (0.06)          (0.06)         (0.02)    
                                                                                                                                  
%% Party size                 -67.81***                 -72.36***           -63.36***      -0.05***        -0.05***       -0.04***   
%%                              (2.00)                   (2.26)              (3.85)         (0.002)         (0.002)        (0.001)   
                                                                                                                                  
%% SMD                          -38.64                   -106.08             -121.47         -0.03           -0.10          -0.03    
%%                             (55.93)                   (65.04)            (115.72)        (0.06)          (0.07)         (0.03)    
                                                                                                                                  
%% SMD x reelect                                        267.50**              3.43                          0.26**         0.11**    
%%                                                      (120.97)            (189.85)                        (0.12)         (0.05)    
                                                                                                                                  
%% Suplente                   -310.36***               -379.09***           -354.13**      -0.35***        -0.36***       -0.35***   
%%                             (110.62)                 (109.03)            (140.76)        (0.11)          (0.11)         (0.05)    
                                                                                                                                  
%% 62nd Leg.                                            689.82***           825.22***                       0.18***        0.08***   
%%                                                       (63.01)             (96.98)                        (0.06)         (0.03)    
                                                                                                                                  
%% 64th Leg.                                             -118.71            523.90***                        -0.09        -0.21***   
%%                                                      (114.65)            (165.78)                        (0.12)         (0.04)    
                                                                                                                                  
%% Extraordinary                                      -1,101.73***        -1,109.15***                                               
%%                                                       (73.24)             (48.29)                                                 
                                                                                                                                  
%% Constant                  3,119.59***               3,122.09***         2,829.41***      5.22***         5.09***        7.26***   
%%                             (70.98)                   (86.15)            (144.04)        (0.14)          (0.16)         (0.08)    
                                                                                                                                  
%% ----------------------------------------------------------------------------------------------------------------------------------
%% Fixed effects                  no                      term                term            no             term           term
%% Random effects                 no                       no                member           no              no             no
%% Estimation method             OLS                      OLS                linear         negative        negative    zero-inflated
%%                                                                        mixed-effects     binomial        binomial     count data   
%% ----------------------------------------------------------------------------------------------------------------------------------
%% Observations                 9,494                     9,494               9,494          9,494           9,494          9,494    
%% R2                            0.15                     0.18                                                                       
%% Adjusted R2                   0.15                     0.17                                                                       
%% Log Likelihood                                                          -85,190.54     -60,825.36      -60,818.74     -55,503.49  
%% theta                                                                                0.16*** (0.002) 0.16*** (0.002)              
%% Akaike Inf. Crit.                                                       170,411.10     121,670.70      121,663.50                 
%% Bayesian Inf. Crit.                                                     170,518.50                                                
%% Residual Std. Error   2,506.13 (df = 9485)     2,467.49 (df = 9481)                                                               
%% F Statistic         208.32*** (df = 8; 9485) 168.54*** (df = 12; 9481)
%% LR test intercept-only      
%% ==================================================================================================================================
%% Note:                                                                                                  *p<0.1; **p<0.05; ***p<0.01
%% \end{verbatim}
%%   \end{tiny}
%%   \caption{Models of legislative debate. Standard errors in parentheses. } 
%%   \label{T:app-regs} 
%% \end{table} 

%\listofendnotes

\bibliographystyle{apsr}

\bibliography{../bib/magar}

%% \begin{thebibliography}{xx}

%% \harvarditem{Alem\'an \harvardand\ Tsebelis}{2016}{aleman-tsebelis-2016-book}
%% Alem\'an, Eduardo \harvardand\ George Tsebelis. 2016.
%% \newblock {\em Legislative Institutions and Lawmaking in Latin America}.
%% \newblock Oxford:  Oxford University Press.

%% \harvarditem{Alem\'an \harvardand\ Navia}{2009}{aleman.navia.UrgChi.2009}
%% Alem\'an, Eduardo \harvardand\ Patricio Navia. 2009.
%% \newblock ``Institutions and the Legislative Success of `Strong' Presidents: An
%%   Analysis of Government Bills in {Chile}.'' {\em Journal of Legislative
%%   Studies} 15(4):401--19.

%% \end{thebibliography}


\end{document}

