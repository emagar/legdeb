
%% To conclude this section, please classify the country along Proksch and Slapin’s (2015) contribution. The authors suggest that there are two extreme poles, depending on rules of procedure and electoral system organization. As described by Proksch and Slapin (2015: 96), in some systems, individual MPs are ‘guaranteed access to speaking time’, and ‘backbenchers are granted equal time as party leaders’. In other systems, the rules severely restrict individuals’ access to the floor and parties draft speakers lists, which gives party leaders much more control on this question. A number of countries fall, however, somewhere between these two extreme categories, giving some opportunity for individual MPs to access the floor, but favoring party lists. Please classify your country accordingly. If your country was included in Proksch and Slapin’s study, please refer to this classification when discussing your country’s institutional setting.

Proksch and Slapin's \citeyearpar{proksch-slapin2015book} scheme, used across chapters in this volume, compares assemblies according to how members gain access to take the floor in order to deliver speeches (p. 79). A continuum connects two extremes: party-controlled access and individual member-controlled access. Formal rules place the Cámara towards the individual member-controlled access limit of the continuum; but partisan rules pull it towards the party-controlled access side. The removal of single term limits ought to make this tension between formal and de facto institutions harder to manage for all parties.  



stands in nearly opposite positions in the continuum when formal or partisan rules 

(a) "parties draw up their own lists and thereby control which members of their party take the floor" and (b) "a nonpartisan figure, usually the Speaker of the House ..., recognizes the right of individual members to speak" (p. 79). 


       P        F
(a) 1  2  3  4  5 (b)

A key question when studying legislative debates is how do members gain access to the plenary floor to deliver speeches; Proksch and Slapin (2015) provide a comparative scheme used elsewhere in this volume where countries are ranked on two extremes: (a) "parties draw up their own lists and thereby control which members of their party take the floor" or (b) "a nonpartisan figure, usually the Speaker of the House ..., recognizes the right of individual members to speak" (p. 79). In this scheme, I classify Mexico as falling, formally, into the latter, but informally,
This categorization accurately characterizes the "unconstrained" speaking time analyzed in the prior research cited above (e.g. Maltzman and Sigelman 1996).


Wrong page!

|                    | low cohesion    | high cohesion   |
| Low personal vote  | same monitoring | same monitoring |
| High personal vote | high monitoring | low monitoring  |


While mixed systems present complex arrangements, simplified in Mx in terms of personal vote incentives. 

Reliance in primaries for SMD candidate selection---mostly by the PAN (Ascencio), on occasions by the PRI (Poire)---opens some exceptions, but in general ballot access is controlled by national and state party leaders. With this feature, systems with SMDs (the most numerous portion of Mexico's mixed system) and closed-list PR (the other portion of the mix) belog at the bottom of Carey and Shugart's \citeyearpar{carey.shugart.1995} ordinal rank of incentives to cultivate personal vote. 

Little intra party heterogeneity makes imperfect control over debate workable. 
But the removal of single-term limits might change this, at least for members with static ambition. If parties retain control over re-nomination, withold it from incumbents could be detrimental. Incumbency advantage is, in part, attributable to competitiveness. Removing of a "prize fighter" \citep{zallerprizeFighters} to secure nomination of a newbie is, at best, risky for the party. The party may wish to end insurgent members but be unable. 

- incentive to be competitive in order to become necessary for party in nomination
- candado partidista.
- dinastías, campeones




