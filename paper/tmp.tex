This chapter describes the institutions of legislative debate in the Mexican Cámara de Diputados and
assesses predictors of floor participation fitting multiple regression models on more than twenty-three thousand speeches between 2006 and 2020. It shows that majority party members get privileged floor access, in both the number of speeches delivered and their word-length, even after accounting for the negative effect of party size. Other status indicators, such as committee chairs, party leaders, and seniority, have more modest positive effects in debate. Women speak more than men. And the removal of single-term limits in 2018, which personalize elections, associate with a significant surge in floor participation. 

with links to personal vote theory

a finding we link to theoretical accounts of legislative competition in personal-
vote-seeking electoral systems.

 

We describe the institutional setting of parliamentary debate in the UK House of Commons and
assess the determinants of participation in Commons’ debates using data on more than two
million speeches from 1979 to 2019. We show that the main determinant of participation in
parliamentary debate in the UK is whether an MP holds an institutionally powerful position in
either the government or opposition parties. In addition, we describe two patterns in the
evolution of debate behaviour in the Commons over time. First, although MPs in government
and opposition leadership positions give more speeches than backbench MPs in all periods that
we study, the speech-making “bonus” these actors enjoy has decreased over time. Second, MPs
have increasingly employed constituency-oriented language in their parliamentary speeches over
the past 40 years; a finding we link to theoretical accounts of legislative competition in personal-
vote-seeking electoral systems.

This chapter analyzes the trends in speaking behavior in the United States Congress from 1921 to
2010 in the House and Senate. We find that key determinants of political behavior from the
existing American and comparative literature (seniority, committee chair, party leadership,
ideological extremism, and majority party membership) correspond to more floor speeches by
members. Senators deliver more speeches per member than their counterparts in the House,
although the determinants of activity are broadly similar. Splitting the results by historical period
and examining the relationship by the polarization of the chamber show that the effect of certain
variables has changed considerably over time. In the House in particular, the effect of committee
chair, extremism and majority party status have increased over time while the effect of seniority
has noticeably decreased in the post-Gingrich period.

