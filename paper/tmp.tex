
%% To conclude this section, please classify the country along Proksch and Slapin’s (2015) contribution. The authors suggest that there are two extreme poles, depending on rules of procedure and electoral system organization. As described by Proksch and Slapin (2015: 96), in some systems, individual MPs are ‘guaranteed access to speaking time’, and ‘backbenchers are granted equal time as party leaders’. In other systems, the rules severely restrict individuals’ access to the floor and parties draft speakers lists, which gives party leaders much more control on this question. A number of countries fall, however, somewhere between these two extreme categories, giving some opportunity for individual MPs to access the floor, but favoring party lists. Please classify your country accordingly. If your country was included in Proksch and Slapin’s study, please refer to this classification when discussing your country’s institutional setting.

Proksch and Slapin's \citeyearpar{proksch-slapin2015book} scheme, used across chapters in this volume, compares assemblies according to how members gain access to take the floor in order to deliver speeches (p. 79). A continuum connects two extremes: party-controlled access and individual member-controlled access. Formal rules place the Cámara towards the individual member-controlled access limit of the continuum; but partisan rules pull it towards the party-controlled access side. The removal of single term limits ought to make this tension between formal and de facto institutions harder to manage for all parties.  



stands in nearly opposite positions in the continuum when formal or partisan rules 

(a) "parties draw up their own lists and thereby control which members of their party take the floor" and (b) "a nonpartisan figure, usually the Speaker of the House ..., recognizes the right of individual members to speak" (p. 79). 


       P        F
(a) 1  2  3  4  5 (b)

A key question when studying legislative debates is how do members gain access to the plenary floor to deliver speeches; Proksch and Slapin (2015) provide a comparative scheme used elsewhere in this volume where countries are ranked on two extremes: (a) "parties draw up their own lists and thereby control which members of their party take the floor" or (b) "a nonpartisan figure, usually the Speaker of the House ..., recognizes the right of individual members to speak" (p. 79). In this scheme, I classify Mexico as falling, formally, into the latter, but informally,
This categorization accurately characterizes the "unconstrained" speaking time analyzed in the prior research cited above (e.g. Maltzman and Sigelman 1996).


Wrong page!

|                    | low cohesion    | high cohesion   |
| Low personal vote  | same monitoring | same monitoring |
| High personal vote | high monitoring | low monitoring  |


While mixed systems present complex arrangements, simplified in Mx in terms of personal vote incentives. 

Reliance in primaries for SMD candidate selection---mostly by the PAN (Ascencio), on occasions by the PRI (Poire)---opens some exceptions, but in general ballot access is controlled by national and state party leaders. With this feature, systems with SMDs (the most numerous portion of Mexico's mixed system) and closed-list PR (the other portion of the mix) belog at the bottom of Carey and Shugart's \citeyearpar{carey.shugart.1995} ordinal rank of incentives to cultivate personal vote. 

Little intra party heterogeneity makes imperfect control over debate workable. 
But the removal of single-term limits might change this, at least for members with static ambition. If parties retain control over re-nomination, withold it from incumbents could be detrimental. Incumbency advantage is, in part, attributable to competitiveness. Removing of a "prize fighter" \citep{zallerprizeFighters} to secure nomination of a newbie is, at best, risky for the party. The party may wish to end insurgent members but be unable. 

- incentive to be competitive in order to become necessary for party in nomination
- candado partidista.
- dinastías, campeones





---------------

To analyze participation in floor debates, I fit multivariate event count models to words spoken. Both specifications of the dependent variable, the absolute and the relative, appear in the left side. In the right side are status variables, member charactersitics, and other controls. Units are member-periods. 

/Status variables/. A dummy for *majority* status indicates members from Morena in the 64th Legislature---the only party controlling over 50 percent of seats. If debate is an (imperfect) substitute for legislative outcomes, then minority members demand more frequent floor participation \citep{proksch-slapin2015book}. On the contrary, if members put value on debate per se, the majority may demand it as much as others, possibly with better access to the floor. Next, a dummy for committee *chair* status. When producing a report, the chair has privileged access to the floor, and this should translate into more speech. A dummy for party *leader* status completes this set. Leaders allocate party speakers. Whether or not they take advantage of this privilege remains an open question, a good leader ought to distribute the goodies, or risk removal. 

/Member variables/. Aside from *woman* and *seniority*, the right side includes *smd*, a dummy equal one for members elected in single-member districts.
The method of election is behind systematic differences in members' pork requests \citep{kerevelPork2015}, which may also translate into higher demand for access to the floor. I also interact this regressor with a dummy indicating the 64th Legislature, which dropped single-term limits (*smd x reelection*). The more personal vote should generate higher demand for floor access. *Party size* is the percentage of seats the member's party holds. Larger parties must divide the slot that all parties get to take the floor among more members, and this should show up as a negative regression coefficient. And a dummy *suplente* controls for substitute members. Regressors not in the right side include members' ages due to incomplete data, and party ideology, which made no difference in the estimates. Replication material is available. 

/Other controls/. Also in the right side are dummies for the *62nd* and *64th* terms (the 60th is the baseline) and another for *extraordinary* periods. Finally, with the option to take leaves of absence, calling in suplentes, some members served incomplete periods. The exposure is the number of days served in the period, logged. Higher exposure offers more opportunities for floor access. 

Negbin vs OLS, zinb and random effects. 


Next is *party size*, the seats that the member's party controlled in the chamber in the term as a proportion of all seats. Since rules allocate speaking time to the parties in proportion to their size, members of smaller parties have more opportunities to speak than  members of larger parties, and the variable should have a negative effect.


The absolute with negbin, the relative *words/exposure* with ols. Variants.

To assess the determinants of participation in legislative debate in a multivariate framework, we estimate a negative binomial regression model in which our outcome variable is the number of speeches made by a given MP in a given Parliament (the period of time between two general elections). We also show results from a linear regression model where our dependent variable captures the number of words spoken by a given MP in a given Parliament.
We describe variation in speaking rates using two sets of variables. First, building on the insights outlined above in our analysis of parliamentary procedures, we include five dummy variables that correspond to important institutional positions that MPs might hold in the House of Commons. As procedural privileges allow frontbenchers to speak more than other members, we include indicators for whether, at any point in the relevant Parliament, an MP was a government junior minister, a cabinet minister, a shadow junior minister, or a shadow cabinet minister. We additionally include a variable which distinguishes those MPs who chair parliamentary committees. The baseline comparison for these variables is any MP who does not hold one of these positions – a “backbench” MP. We also include a binary variable to capture differences between MPs from the governing party (or parties) and MPs from non-governing parties.
Aside from these institutional variables, we include a series of MP-specific characteristics. In particular, we include an indicator for whether the MP is female, the age of the MP (in years), age squared, and the seniority of the MP (in years). To evaluate whether MPs speech-making behaviour is sensitive to electoral security – a hypothesis advanced elsewhere (Kellermann 2016) – we also include the margin of victory of the MP in the most recent general election (in percentage points). We also include a set of dummy variables to capture the effects of party affiliation as well as fixed-effects for each parliamentary term. Finally, we also control for a continuous variable (“Exposure”) that measures the percentage of time in which the MP held her seat during a given Parliament (we include the logged version of this variable in the negative binomial model).
We include the same sets of covariates for both of our outcome variables, and the results of our regressions are presented in table 3, and we illustrate the coefficients associated with the key explanatory variables in figures 3 and 4.


