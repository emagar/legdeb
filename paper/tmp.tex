
Le adjunto las hojas de excel con los cambios en presidencias. Hay un par de cosas que me llamaron la atención: 

1. En la Legislatura LX, parece que se divide a la Comisión de Justicia y Derechos Humanos en dos. Para la comisión de Justicia permanece la misma persona pero, hay haber dos comisiones se asigna a una presidenta nueva. No sé si esto pueda considerarse como un cambio. 

2. En la Legislatura LXIV se pide consideración y parece (pero no estoy segura) que se somete a votación la presidencia de la comisión de Asuntos de la Frontera Sur, sin embargo, no encuentro las votaciones o resolución de esta moción. 

3. En la Legislatura LXIV, hay una baja en la presidencia de la comisión de Radio y Televisiones, pero no está el nombre de la persona que sustituye.

-------

A prominent study of agenda power in Mexico aims focus on the executive in the legislative arena, and only secondarily on intra-cameral institutions. \citet{casar.agsetting.2016} characterizes debate in Congress as centralized: "[governing] bodies have the power ... to conduct floor debates, including assigning turns and time to speakers" (p. 154). The review of debate rules I conduct in this section reveals that the author must have /de facto/ practice in mind: formal rules actually decentralize agenda power to a considerable extent. 

positive agenda power in mind---formal rules preclude negative 


This, we will see, comes from party discipline, because formal institutions establish individual member rights to be recognized by the presiding officer.


No systematic study of debate rules.

Claims J and M have (more) agenda power (than others).

Tacit: talks of positive agenda power (see mccox), if negative considered, they don't have it. Others have formal positive power too, so leaders don't have negative power.

True due to party discipline. 



\citet{casar.agsetting.2016} characterizes debate as party-centered: "[governing] bodies have the power ... to conduct floor debates, including assigning turns and time to speakers" (p. 154). This, we will see, comes from party discipline, because formal institutions establish individual member rights to be recognized by the presiding officer.

%\citet{casar.agsetting.2016} examination of agenda setting puts the focus on results (passage of legislation). Her mention to debate characterizes it as party-centered: "[governing] bodies have the power ... to conduct floor debates, including assigning turns and time to speakers" (p. 154). This, we will see, is not in alignment with formal institutions, which establish individual member rights to be recognized by the presiding officer.



-------------------------------

for section 2


%\subsection{Legislative parties}

Weak parties in the electorate lie in sharp contrast to strong legislative parties, which they draw from electoral rules. The formula is mixed member plurality---three-hundred deputies are elected every three years by first-past-the-post in single member districts (SMDs), two-hundred more by closed-list proportional representation (PR), all seats contested in races concurrent with the presidential election, then again at the presidential midterm \citep{weldonMixedMemberSys2001}.

What gave leaders their centrality were two other key features. Single-term limits, which the constitution set on every elected officeholder, diverted all political ambition to the progressive format \citep{schlesinger.1966}. And centralized ballot access gives national and state party leaders control of future political careers \citep{langston.2008}.\footnote{Reliance in primaries for SMD candidate selection, mostly by the PAN \citep{ascencio.kerevel.cand-sel-beh.2021}, on occasions by the PRI \citep{poire.phd.2002}, opens room for exceptions to centralized ballot access. They deserve closer attention.}

Leaders control a stock of selective incentives to reward loyalty. Leaders distribute their party's share of committee chairs and seats. The Junta appoints members at the start of the term, and freely makes replacements afterwards by simple announcement to the floor. This is a key selective incentive to achieve collective action in the partisan theory of congressional organization \citep{cox.mccubbins.1993}. Leaders have other carrots and sticks in the form of discretionary spending. By one count, leaders of the 60th Legislature (2006-2009, included in the data) routinely received discretionary spending authority over one-fifth of the Cámara's yearly budget---plane tickets, bonus payments, and income tax breaks that could be handed to the rank and file \citep{casar.2011}.

This institutional combination both removes personal vote incentives \citep{carey.shugart.1995,cain.etal.1987} and rewards top-bottom discipline. An indicator is cohesion, which is near perfect across parties. \citet{tellez-del-rio.2018} computed frequencies with which deputies voted against a majority of their party. Excluding unanimous votes, the mean for the 1997--2018 period is just 2 percent, or 3.4 percent when abstentions are counted as votes against the party majority (p. 25). 

Discipline plays a fundamental role in floor access. Formal rules, we see next, make it very difficult to control the flow of legislation without legislative parties. Party discipline operates as an alternative to agenda cartelization in many systems \citep{prata.2006}, including the Cámara. 




for intro to fer gaby lupita

%\subsection{Party discipline as alternative to centralized agenda power}

Rules like these are ill-designed to prevent plenary bottlenecks \citep{cox.2006}. Even in the presence of a majority party, individual members retain speaking rights that water down the Junta's efforts to cartelize the legislative process. Absent formidable party discipline, preventing dilatory tactics would be enormously difficult. The final section elaborates. 

