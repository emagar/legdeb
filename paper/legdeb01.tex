\documentclass[letter,12pt]{article}
\usepackage[letterpaper,right=1in,left=1in,top=1in,bottom=1in]{geometry}
\usepackage{setspace}

\usepackage[utf8]{inputenc}   % allows input of special characters from keyboard (input encoding)
\usepackage[T1]{fontenc}      % what fonts to use when printing characters       (output encoding)
\usepackage{amsmath}          % facilitates writing math formulas and improves the typographical quality of their output
\usepackage[hyphens]{url}     % adds line breaks to long urls
\usepackage[pdftex]{graphicx} % enhanced support for graphics
\usepackage{tikz}             % Easier syntax to draw pgf files (invokes pgf automatically)
\usetikzlibrary{arrows}

\usepackage{mathptmx}           % set font type to Times
\usepackage[scaled=.90]{helvet} % set font type to Times (Helvetica for some special characters)
\usepackage{courier}            % set font type to Times (Courier for other special characters)

\usepackage[longnamesfirst, sort]{natbib}\bibpunct[]{(}{)}{,}{a}{}{;} % handles biblio and references 

\usepackage{rotating}         % sideway tables and figures that take a full page
\usepackage{caption}          % allows multipage figures and tables with same caption (\ContinuedFloat)

\usepackage{dcolumn}          % needed for apsrtable and stargazer tables from R to compile
\usepackage{arydshln}         % dashed lines in tables (hdashline, cdashline{3-4}, 
                              %see http://tex.stackexchange.com/questions/20140/can-a-table-include-a-horizontal-dashed-line)
                              % must be loaded AFTER dcolumn, 
                              %see http://tex.stackexchange.com/questions/12672/which-tabular-packages-do-which-tasks-and-which-packages-conflict


\newcommand{\mc}{\multicolumn}

%% TO ADD NOTES IN TEXT, PUT % BEFORE THE ONE YOU WANT DISABLED
\usepackage[disable]{todonotes}                            % no show
%\usepackage[colorinlistoftodos, textsize=small]{todonotes} % show notes
\newcommand{\emm}[1]{\todo[color=red!15, inline]{\textbf{Eric:} #1}}
\newcommand{\vp}[1]{\todo[color=green!15, inline]{\textbf{Vale:} #1}}
\newcommand{\ges}[1]{\todo[color=blue!15, inline]{\textbf{Ges:} #1}}

\usepackage{xr} % allows cross-ref to other file
\externaldocument{urge15appendix}

%% %for submission: sends figs, tables, and footnotes to last pages
%% \RequirePackage[nomarkers,nolists]{endfloat}     % sends tables and figures to the end
%% \RequirePackage{endnotes}                        % turns fn into endnotes; place \listofendnotes where you want 
%%                                                  %the endnotes to appear (it must be after the last endnote).
%% \let\footnote=\endnote
%% \newcommand{\listofendnotes}{
%%    \begingroup
%%    \parindent 0pt
%%    \parskip 2ex
%%    \def\enotesize{\normalsize}
%%    \theendnotes
%%    \endgroup
%% }
%% 
%% % for submission: drop page numbers when producing title page
%% \pagenumbering{gobble} % Remove page numbers (and reset to 1)
%% \pagenumbering{arabic}% Arabic page numbers (and reset to 1)


% agradecimientos
%% Vidal Mendoza RA
%% Eugenio Solís RA
%% Sonia Kuri RA
%% Ana Lu RU

\setcitestyle{citesep={;}}

\usepackage{listings}

\begin{document}

\title{Speach in Mexico's Cámara de Diputados}
\author{Eric Magar \\ Instituto Tecnológico Autónomo de México}
\date{\today}
\maketitle

\newpage

\begin{abstract}
\noindent Text as data: speaches in lower chamber of Mexico's federal Congress. Analysis covers three pre-midterm election legislative terms since 2006. Argument, findings.\footnote{{Data and supporting materials necessary to reproduce the numerical results in the article are available in the following repository (\url{https://github.com/emagar/legdeb}). Supplementary material for this article is available in the appendix in the online edition.}}
\newline
\newline
\textbf{Keywords}: Speach, Congress, presidentialism, Mexico 
\end{abstract}

\newpage

\doublespacing

\section{Introduction}

\section{Paste in paper}

\subsection{parties in camara}

\singlespacing
\begin{footnotesize}
\begin{verbatim}
|----------------------------+------+------+--------|
|                            | 60th | 62nd |   64th |
| party                      |    % |    % |      % |
|----------------------------+------+------+--------|
| pan                        |   41 |   23 |     16 |
| pri                        |   21 |   43 |      9 |
| prd                        |   25 |   20 |      4 |
| morena                     |      |      |     51 |
| opportunistic w/ president |      |    8 |     14 |
| other opportunistic        |   13 |    7 |      6 |
|----------------------------+------+------+--------|
| Total                    % |  100 |  100 |    100 |
|                          N |  500 |  500 |    500 |
|----------------------------+------+------+--------|
| president's party          |  pan |  pri | morena |
|----------------------------+------+------+--------|
\end{verbatim}
\end{footnotesize}
\doublespacing

%% |              |  60 |  62 |     64 |
%% |              |   N |   N |      N |
%% |--------------+-----+-----+--------+
%% | pan          | 207 | 114 |     79 |
%% | pri          | 104 | 213 |     47 |
%% | prd          | 125 | 102 |     20 |
%% | morena       |     |     |    255 |
%% | opport w pdt |     |  38 |     69 |
%% | opport/oth   |  64 |  33 |     30 |
%% |--------------+-----+-----+--------+
%% | Total        | 500 | 500 |    500 |


\subsection{Terminology}

%DONE 3. In terms of window of observation/time period under study: we don’t have a particular guideline for this. Please use the window of observation that you believe is more representative of the politics of legislative debate in your country. Ideally we would like each chapter to include several legislative periods, but we are pragmatic here, considering data availability.

- A *Legislature* is an elected chamber for a legislative term, called a Congress in the U.S. Concurrent with presidential elections the chamber of deputies renovates in whole, and again at the presidential mid-term. Diputados remain three years in office and were single term-limited up to 2021. The 2021 mid-term election will be the first since 1932 to allow incumbents on the ballot, a major change in Mexican legislative politics. Analysis includes the 60th, 62nd, and 64th Legislatures (the Mexican Congress relies on Roman numerals to distinguish Legislatures since the second half of the Nineteenth century).

- Legislative years break into two *ordinary legislative periods*, one covering the months of September through December, inclusive, another February through April, also inclusive. *Extraordinary legislative periods* may be convened during the recess in order to consider a specific bill. Analysis aggregates each member's speeches in the duration of a given period (merging together all extraordinary periods that year, if any). So members in a legislative year like 2012-13 (that had no extraordinary periods) have two word aggregates in the dataset, one for each ordinary period; in a year like 2013-14 (that did), they have three word aggregates in the data. Periods are the units of observation in the analysis. 

- A *plenary session* (or simply a session) is a specific date in the calendar when diputados met. During ordinary periods, sessions are usually held on Tuesdays and Thursdays, and may be scheduled in other weekdays if the Jucopo so decides. Diputados met on forty and thirty-one days in the first and second ordinary periods of 2013-14, respectively, and nine days in extraordinary periods, for a yearly total of eighty session days. (A session in North-American legislative parlance is a Mexican period.)

\subsection{The dependent variables}

As in the other chapters, the dependent variable is the number of words that members spoke in the chamber. A given diputado's words throughout a plenary session were summed into a daily total. Daily total less than 50 words were arbitrarily interpreted as not constituting a speach and removed from the data (ie., the member received a value of zero words that day). Thus filtered, members' daily totals were added across sessions in the same period, producing word aggregates for analysis. Figure XXX portrays speech distributions across periods. Solid points report periods' median speech length in words. It is quite clear in the plot that, with few exceptions, period medians are much the same as the overall median speech length of 599 words. Mild legislatura effects show up too, the 60th medians slightly above and the 64th slightly below the overall median. Horizontal lines report the spread of the central portion of the density---the thicker line is the inter-quartile range, the thinner connects the first and ninth deciles. Period distributions are, in general, similar. The clearest exceptions are extraordinary periods, drawn in gray. 

\begin{figure}
  \centering
    \includegraphics[width=.8\columnwidth]{../plots/quantiles-periodo.pdf}
    \caption{Speeches in the legislative period observed (presiding officers excluded). Solid points indicate the median speech length in the period. Thick and thin lines connect the 25--75 and 10--90 percentiles, respectively. Hollow points are minima and maxima. Ordinary periods in black, extraordinary periods in gray.}\label{F:quantiles}
\end{figure}

%% \singlespacing
%% \begin{footnotesize}
%% \begin{verbatim}
%% Members' daily number of words (percentiles)
%% | periodo      | min | 10% | 25% | 50% |  75% |  90% |   max |
%% |--------------+-----+-----+-----+-----+------+------+-------|
%% | 60y1-1       |  50 | 153 | 436 | 666 |  853 | 1169 |  3057 |
%% | 60y1-2       |  51 | 141 | 380 | 655 |  955 | 1221 |  3434 |
%% | 60y2-1       |  51 | 162 | 407 | 665 |  937 | 1250 |  3650 |
%% | 60y2-2       |  50 |  96 | 256 | 571 |  824 | 1059 | 15932 |
%% | 60y3-1       |  50 | 143 | 407 | 647 |  911 | 1291 |  6193 |
%% | 60y3-2       |  53 | 116 | 423 | 682 |  938 | 1278 |  3248 |
%% | 62y1-1       |  50 | 234 | 447 | 620 |  817 | 1137 |  4700 |
%% | 62y1-2       |  50 | 170 | 451 | 608 |  836 | 1190 |  6353 |
%% | 62y1-extra   | 186 | 433 | 553 | 758 | 1103 | 1512 |  3310 |
%% | 62y2-1       |  50 | 149 | 438 | 620 |  876 | 1293 |  7367 |
%% | 62y2-2       |  56 | 186 | 396 | 599 |  795 | 1159 |  5704 |
%% | 62y2-extra   |  50 | 410 | 543 | 751 | 1147 | 1831 |  9765 |
%% | 62y3-1       |  56 | 189 | 428 | 610 |  851 | 1366 |  4356 |
%% | 62y3-2       |  55 | 202 | 379 | 537 |  733 | 1032 |  4517 |
%% | 64y1-1       |  50 | 136 | 278 | 506 |  703 |  892 |  4201 |
%% | 64y1-2       |  51 | 158 | 402 | 556 |  706 |  989 |  3085 |
%% | 64y1-extra   |  57 | 370 | 545 | 670 |  955 | 1317 |  2957 |
%% | 64y2-1       |  50 | 143 | 351 | 585 |  772 | 1060 |  6358 |
%% | 64y2-2       |  51 | 129 | 186 | 498 |  702 |  881 |  3112 |
%% \end{verbatim}
%% \end{footnotesize}
%% \doublespacing

Hollow points are the periods' minima and maxima. Diputada Valentina Batres holds the record for delivering the longest speech in the three Legislaturas scrutinized. At 15,932 words, the speech delivered March 11th, 2008 is 50 percent longer than the runner-up and has about as many words as \emph{Don Quijote de la Mancha}'s chapters 1 through 7 (forty-five pages in the edition I own). Batres and legislators close to AMLO used dilatory tactics throughout the session, filibustering consideration of the new Sistema Nacional de Información Estadística y Geográfica. 
  
%Deputies close to AMLO were adoting dilatory tactics. Batres requested the addition of a point to JuCoPo's order of the day. Mesa Directiva denied, so a PRD faction threatened to take over the Tribuna (?). Batres introduced motion to suspend and other dilatory tactics (ley sis nal informacion inegi), then filibustered (called art. 103 ley reglamentaria, granting her 30 minutes to present minority vote despite JuCoPo's day aggrrement to limit to 10 minutes.)

% Chapters 1 through 6 of El Quijote 13,049, 40 pages in my edition.
% Chapters 1 through 7 of El Quijote 14,916, 45 pages in my edition.
% Sonnets and Chapters 1 through 7 of El Quijote 16,191, 45+ pages in my edition.
% Chapters 1 through 7 of El Quijote 17,912, 45+ pages in my edition.

The log scale helps appreciate the low side of the distribution. But it also blurs subtle but revealing differences in the high side. From 60th to 62nd, max went up while percentiles 75 and 90 remain at similar levels. Like Batres, an unusually high top decile consists of attempts to disrupt debate through filibustering. Dilatory tactics went down in 62nd compared with 60th, and sunstantially so in the 64th with a single party majority.

\singlespacing
\begin{footnotesize}
\begin{verbatim}
Words per day
| Legislatura    | min | 10% | 25% | 50% | 75% |  90% |   max |
| 60th           |  50 | 137 | 392 | 652 | 901 | 1215 | 15932 |
| 62nd           |  50 | 193 | 438 | 611 | 850 | 1254 |  9765 |
| 64th (partial) |  50 | 142 | 327 | 547 | 730 |  975 |  6358 |
\end{verbatim}
\end{footnotesize}
\doublespacing

The statistics above all exclude presiding officers. Presiding officers speak routinely each session but do not engage in debate. One is the chamber president, an officer similar to the Speaker in the UK House of Commons. Diputados are expected to address the chamber president, who is responsible for keeping debate orderly and within chamber and congressional rules. Other officers are secretarios, responsible for formalities such as reading bills, committee reports, or other motions presented to the plenary for consideration, announcing the result of roll call votes, an so forth.  

\singlespacing
\begin{footnotesize}
\begin{verbatim}
| Members in office                                       |   % |    N |
|---------------------------------------------------------+-----+------|
| one observed Legislatura only                           |  86 | 1470 |
| one observed and one or more unobserved Legislaturas    |   9 |  149 |
| two observed (and zero or more unobserved) Legislaturas |   5 |   91 |
| three observed Legislaturas                             |   0 |    0 |
|---------------------------------------------------------+-----+------|
| Total                                                   | 100 | 1710 |
\end{verbatim}
\end{footnotesize}
\doublespacing

The table offers contrast between diputados and presiding officers, adopting members as units instead of daily sessions, as above. Speechless members, who remain unobserved in session-level data, are now accounted for. Of 1,710 members observed across the three Legislaturas, 156 never uttered a single word in the plenary throughout their tenure. The median member spoke 3,677 words in total, the median presiding officer spoke 46,710. Since tenure lengths vary (91 members, or 5 percent of all, were present in two of the Legislaturas observed), a relative measure is desirable. The median diputado spoke 26.4 words per session on average, while the median presiding officer spoke an average 258.6 words per session. Presiding officers are therefore dropped from the analysis. 

95 percent of members never served as presiding officers. Of those who did, the mode was 

\singlespacing
\begin{footnotesize}
\begin{verbatim}
| Years as presiding officer |   % |    N |
|----------------------------+-----+------|
| never                      |  95 | 1630 |
| one                        |   3 |   49 |
| two                        |   1 |   19 |
| three                      |   1 |   12 |
|----------------------------+-----+------|
|                            | 100 | 1710 |
\end{verbatim}
\end{footnotesize}
\doublespacing


\singlespacing
\begin{footnotesize}
\begin{verbatim}
* * Member descriptives for diputados vs presiding officers * *
|                       |  min |   25% |   50% |   75% |    max |
|-----------------------+------+-------+-------+-------+--------|
| DIPUTADOS             |      |       |       |       |        |
| Total words in period |    0 |  1128 |  3677 |  8535 | 242434 |
| Sessions in office    |    1 |   121 |   180 |   201 |    522 |
| Words by session      |    0 |   9.9 |  26.4 |  57.4 | 1212.2 |
|-----------------------+------+-------+-------+-------+--------|
| PRESIDING OFFICERS    |      |       |       |       |        |
| Total words in period | 3141 | 25680 | 46710 | 71669 | 320155 |
| Sessions in office    |   32 |   121 |   182 |   219 |    522 |
| Words by session      | 19.6 | 137.7 | 258.6 | 436.5 | 3563.3 |

* * Member descriptives for diputados vs presiding officers * *
|                       |  min |   10% |   25% |   50% |   75% |    90% |    max |
|-----------------------+------+-------+-------+-------+-------+--------+--------|
| DIPUTADOS             |      |       |       |       |       |        |        |
| Total words in period |    0 |   135 |  1128 |  3677 |  8535 |  16948 | 242434 |
| Sessions in office    |    1 |    25 |   121 |   180 |   201 |    219 |    522 |
| Words by session      |    0 |   1.1 |   9.9 |  26.4 |  57.4 |  109.2 | 1212.2 |
|-----------------------+------+-------+-------+-------+-------+--------+--------|
| PRESIDING OFFICERS    |      |       |       |       |       |        |        |
| Total words in period | 3141 | 13046 | 25680 | 46710 | 71669 | 148511 | 320155 |
| Sessions in office    |   32 |   121 |   121 |   182 |   219 |    227 |    522 |
| Words by session      | 19.6 |  80.9 | 137.7 | 258.6 | 436.5 |  802.0 | 3563.3 |

\end{verbatim}
\end{footnotesize}
\doublespacing


\subsection{Justify pre-midterm selction}

Exclusion of 61 and 63 was unintentional, time constraints.

But there are good reasons why attention to this sample ok.

Could not go too far anyway (57 58 59 60 61 62 63 64 since democratization). 60 62 64 covers good part.

64 incomplete. Included because first instance of single-party unified government. And, importantly, first Congress where incumbents allowed to be on the ballot in midterm. Plus clear mutation of governing groups after 2+ decades of the partidocracia party system \citep{magar.2007ref.2015,magar.estevez.rosas.2010}. 

Pre-midterm with small N offers control, to the extent that pre- and post-midterm election are different.



\subsection{N speakers}

\begin{figure}
  \centering
    \includegraphics[width=.8\columnwidth]{../plots/nspeakers.pdf}
    \caption{How many spoke in the plenary? Daily plenary sessions of the 60th, the 62nd, and the first two years of the 64th Legislaturas.}\label{F:nspeakers}
\end{figure}





Two DV specifications. 


\section{Multivariate analysis}

The dependent variable is how much a member spoke in plenary debates in a legislative period. Two specifications are analyzed, one absolute, one relative. The absolute *words_(i,p)* is member i's total words in period p. Days where member i spoke less than 50 words in total are considered non-debates, adding zero towards the period's aggregate. The relative specification is *words_(i,p)* divided by the session days that member i served in office as a proportion of all session days in period p. So the denominator for a member who spoke 5 thousand words and served uninterrupted throughtout period p is one, and the two specifications equal 5 thousand. If the member took a leave of absence and served only half of the period the denominator is 0.5, making the relative words equal 5000/0.5=10000.  

The right side of the equation includes explanatory variables in three groups: the member's party, the member's position in the chamber hierarchy, and the member's personal traits. The partisan group includes *majority*, a dummy equal one if the member belonged to the majority party, zero otherwise. Given that one term only had a majority party, the variable indicates Morena party deputies in the 64th Legislature. Expectation. Next is *party size*, the seats that the member's party controlled in the chamber in the term as a proportion of all seats. Since rules allocate speaking time to the parties in proportion to their size, members of smaller parties have more opportunities to speak than  members of larger parties, and the variable should have a negative effect.

A pair of dummies controls for the party's ideology. *Right* equals 1 for members of the right-of-center PAN, 0 otherwise. Left equals 1 for PRD members in 60th and 62nd, and Morena member in the 64th, 0 otherwise. The omitted group includes members of the PRI and the smaller opportunistic parties. The dummy should capture any systematic effect of left's more frequent filibustering attempts. (There is no a priori expectation associated with left and right.)

The chamber hierarchy group includes three explanatory variables. Seniority measures the terms as federal deputy that member i served befre the current term. With single-term limits, members were only eligible to run as diputados for non-consecutive terms, something few did. There are members of two types only in the data, those  

hierarchy

seniority
leader
chair


member

female
(age)
smd
sup


\begin{footnotesize}
  
% Table created by stargazer v.5.2.2 by Marek Hlavac, Harvard University. E-mail: hlavac at fas.harvard.edu
% Date and time: Thu, Jun 11, 2020 - 02:58:25 PM
% Requires LaTeX packages: dcolumn 
\begin{table}[!htbp] \centering 
  \caption{Regression results} 
  \label{} 
\begin{tabular}{@{\extracolsep{5pt}}lD{.}{.}{-2} D{.}{.}{-2} D{.}{.}{-2} D{.}{.}{-2} D{.}{.}{-2} } 
\\[-1.8ex]\hline 
\hline \\[-1.8ex] 
 & \multicolumn{5}{c}{\textit{Dependent variable:}} \\ 
\cline{2-6} 
\\[-1.8ex] & \multicolumn{2}{c}{Words in period} & \multicolumn{3}{c}{Words in period (relative to tenure)} \\ 
\\[-1.8ex] & \multicolumn{2}{c}{\textit{negative}} & \multicolumn{2}{c}{\textit{OLS}} & \multicolumn{1}{c}{\textit{linear}} \\ 
 & \multicolumn{2}{c}{\textit{binomial}} & \multicolumn{2}{c}{\textit{}} & \multicolumn{1}{c}{\textit{mixed-effects}} \\ 
\\[-1.8ex] & \multicolumn{1}{c}{(1)} & \multicolumn{1}{c}{(2)} & \multicolumn{1}{c}{(3)} & \multicolumn{1}{c}{(4)} & \multicolumn{1}{c}{(5)}\\ 
\hline \\[-1.8ex] 
 Tenure (logged) & 0.94^{***} & 0.97^{***} &  &  &  \\ 
  & (0.04) & (0.04) &  &  &  \\ 
  & & & & & \\ 
 Majority & 0.73^{***} & 0.67^{***} & 4,476.27^{***} & 5,258.27^{***} & 4,145.80^{***} \\ 
  & (0.11) & (0.17) & (292.98) & (351.82) & (562.84) \\ 
  & & & & & \\ 
 Party size & -5.10^{***} & -5.04^{***} &  &  &  \\ 
  & (0.21) & (0.24) &  &  &  \\ 
  & & & & & \\ 
 Right &  &  & -1,162.20^{***} & -1,546.63^{***} & -1,359.04^{***} \\ 
  &  &  & (59.95) & (71.42) & (117.24) \\ 
  & & & & & \\ 
 Left & 0.01 & 0.09 & -65.14 & 94.10 & 81.06 \\ 
  & (0.07) & (0.07) & (77.06) & (77.01) & (143.66) \\ 
  & & & & & \\ 
 Seniority & 0.28^{***} & 0.33^{***} & 12.87 & 135.88^{*} & 156.55 \\ 
  & (0.08) & (0.08) & (77.05) & (78.17) & (137.38) \\ 
  & & & & & \\ 
 Party leader & 0.17^{***} & 0.18^{***} &  &  &  \\ 
  & (0.05) & (0.05) &  &  &  \\ 
  & & & & & \\ 
 Comm. chair &  &  & 363.47^{***} & 287.16^{***} & 435.96^{***} \\ 
  &  &  & (83.26) & (82.49) & (117.33) \\ 
  & & & & & \\ 
 SMD & 0.42^{**} & 0.40^{*} & 2,316.93^{***} & 2,386.04^{***} & 1,469.52^{***} \\ 
  & (0.21) & (0.21) & (210.89) & (208.99) & (319.39) \\ 
  & & & & & \\ 
 Suplente & 0.35^{***} & 0.32^{***} & -86.78 & 104.10 & 64.28 \\ 
  & (0.09) & (0.09) & (92.38) & (92.10) & (155.50) \\ 
  & & & & & \\ 
 62nd Leg. & -0.04 & -0.04 & -294.59^{***} & -186.99^{***} & -235.74^{**} \\ 
  & (0.06) & (0.06) & (56.81) & (56.81) & (98.74) \\ 
  & & & & & \\ 
 64th Leg. & -0.34^{***} & -0.33^{***} & -464.64^{***} & -388.86^{***} & -287.62^{**} \\ 
  & (0.11) & (0.11) & (113.82) & (112.58) & (144.75) \\ 
  & & & & & \\ 
 Female &  & 0.23^{***} &  & -691.36^{***} & -484.89^{***} \\ 
  &  & (0.06) &  & (76.88) & (122.93) \\ 
  & & & & & \\ 
 Constant &  & 0.18^{*} &  & 635.84^{***} & 911.97^{***} \\ 
  &  & (0.10) &  & (94.29) & (145.71) \\ 
  & & & & & \\ 
 dfem & 0.08 & 0.06 & 98.66^{*} & 114.92^{**} & 9.78 \\ 
  & (0.06) & (0.06) & (56.66) & (56.59) & (102.72) \\ 
  & & & & & \\ 
 Constant & 2.57^{***} & 2.31^{***} & 2,720.24^{***} & 3,180.82^{***} & 2,946.45^{***} \\ 
  & (0.15) & (0.17) & (75.42) & (105.36) & (166.74) \\ 
  & & & & & \\ 
\hline \\[-1.8ex] 
Observations & \multicolumn{1}{c}{9,494} & \multicolumn{1}{c}{9,494} & \multicolumn{1}{c}{9,494} & \multicolumn{1}{c}{9,494} & \multicolumn{1}{c}{9,494} \\ 
R$^{2}$ &  &  & \multicolumn{1}{c}{0.10} & \multicolumn{1}{c}{0.12} &  \\ 
Adjusted R$^{2}$ &  &  & \multicolumn{1}{c}{0.10} & \multicolumn{1}{c}{0.12} &  \\ 
Log Likelihood & \multicolumn{1}{c}{-60,817.31} & \multicolumn{1}{c}{-60,811.57} &  &  & \multicolumn{1}{c}{-85,479.08} \\ 
$\theta$ & \multicolumn{1}{c}{0.16^{***}  (0.002)} & \multicolumn{1}{c}{0.16^{***}  (0.002)} &  &  &  \\ 
Akaike Inf. Crit. & \multicolumn{1}{c}{121,658.60} & \multicolumn{1}{c}{121,651.10} &  &  & \multicolumn{1}{c}{170,988.20} \\ 
Bayesian Inf. Crit. &  &  &  &  & \multicolumn{1}{c}{171,095.50} \\ 
Residual Std. Error &  &  & \multicolumn{1}{c}{2,578.64 (df = 9483)} & \multicolumn{1}{c}{2,547.33 (df = 9481)} &  \\ 
F Statistic &  &  & \multicolumn{1}{c}{105.02$^{***}$ (df = 10; 9483)} & \multicolumn{1}{c}{109.40$^{***}$ (df = 12; 9481)} &  \\ 
\hline 
\hline \\[-1.8ex] 
\textit{Note:}  & \multicolumn{5}{r}{$^{*}$p$<$0.1; $^{**}$p$<$0.05; $^{***}$p$<$0.01} \\ 
\end{tabular} 
\end{table} 

\end{footnotesize}

\begin{figure}
  \centering
    \includegraphics[width=.8\columnwidth]{../plots/avgMgEffects.pdf}
    \caption{Average marginal effects from model 5. Dots report how the probability of an urgent bill
changes in response to a unit change in each independent variable, all else at mean values; bars are
95-percent confidence intervals.}\label{F:avgmgeff}
\end{figure}

Figure \ref{F:avgmgeff} reports changes in the average predicted number of words in the period associated with unit changes in model 5's explanatory variables (all other regressors at their mean value). This is a convenient way to gauge negative binomial regression coefficients, by translating them into interpretable quantities.
The report from a committee with a coalition chair experiences a 0.17
hike (0.06 standard error) in the likelihood of receiving a closed rule compared to a report by an
opposition-chaired committee. The effect is as big as the average marginal effects of Hacienda
Referral (0.18), which capures mostly high-significance draft laws, and that of Multiple Referrals
(0.16), which we view as an indicator of issue complexity. We therefore find no statistical evidence
to reject our Hypothesis 1. The results also confirm hypothesis 2, showing that a bill reported by a
generally less friendly committe (chaired by the opposition), has a higher probability of receiving
an open rule on the floor, thereby allowing the floor majority to bring back the bill to the median
through floor amendments. Thus, presidents use open rules to control bills coming from preference
distant committee chairs.
The substantial effects of Hacienda Referral and Multiple Referrals deserve comment. They
suggest, first, that when spending gets in the way, restrictive rules are the norm in Chile. Recall
that Multiple Referrals exclude the Finance Committee, so there is an independent effect of bills
with jurisdictional overlaps worth investigating further, and which must be associated, in part at
least, to influencing the report through a friendlier committee. 16 Furthermore, note that the Finance
Committee was always chaired by a coalition member but, with the exception of the 1998–2000
period, never by a co-partisan of the president. This may explain the milder effect of the partisan
specification of our key variable in model 1.
Another effect worth highlighting is Introd. in Senate. Bills successfully passing the Upper
Chamber first, where the opposition was systematically larger and attimes in control, were much
less likely to get urgent status (the average marginal effect is −0.15 and significant). This suggests
that agreements and compromises reached in the Senate ignited less, not more, protection from
floor amendments in the Cámara’s plenary, most likely as a consequence of the greater prefer-
ence divergence between the President and the opposition-led Senate. Analysis of inter-chamber
negotiation and the reliance on urgency in the Upper Chamber are interesting venues for future
research.
Finally, there are time trends in fast-track authority that simulations reveal neatly. Figure 5
portrays the predicted probability that a bill enters the fast-track throughout the legislative year.
Regressors in model 3 are held constant to simulate a bill sent to the Cámara in the 2006–10
Legislature that was referred to a single committee, excluding Hacienda. Presidential Approval
(insignificant across models) is set to the mean for President Bachelet’s first term, coinciding in full
with the 2006–10 Legislature. The inverted-U shape shows how fast-track probability, predicted
at 0.17 for coalition-chaired committees at the start, and 0.08 for the rest, becomes much likelier
in the first half of the legislative year. By the second quarter (June–August), the probability is at
its peak, about 0.32 percent and 0.17, respectively. It then experiences a sharp drop, ending the
austral Summer break at 0.13 for coalition-chaired committees, and 0.05 for others. And while 95-
percent confidence bands overlap, they barely do so at the middle of the legislative year, lending
confidence that we are picking up a signal and not just random noise.


\section*{Acknowledgements}
Eric Magar received financial support from the Asociaci\'on Mexicana de Cultura \textsc{a.c.}\ and \textsc{conacyt}'s Sistema Nacional de Investigadores. The author is grateful to xxx for research assistance.  The author are responsible for mistakes and shortcomings in the study.

\listofendnotes

\bibliographystyle{apsr}

\bibliography{../bib/magar}

%% \begin{thebibliography}{xx}

%% \harvarditem{Alem\'an \harvardand\ Tsebelis}{2016}{aleman-tsebelis-2016-book}
%% Alem\'an, Eduardo \harvardand\ George Tsebelis. 2016.
%% \newblock {\em Legislative Institutions and Lawmaking in Latin America}.
%% \newblock Oxford:  Oxford University Press.

%% \harvarditem{Alem\'an \harvardand\ Navia}{2009}{aleman.navia.UrgChi.2009}
%% Alem\'an, Eduardo \harvardand\ Patricio Navia. 2009.
%% \newblock ``Institutions and the Legislative Success of `Strong' Presidents: An
%%   Analysis of Government Bills in {Chile}.'' {\em Journal of Legislative
%%   Studies} 15(4):401--19.

%% \end{thebibliography}


\end{document}

